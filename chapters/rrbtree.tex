\chapter{Relaxed Radix Balanced Tree}

\section{Relaxed Radix Search}

When \rbtree\ is relaxed, it is no longer possible to compute indices from a search key without additional metadata in the form of \emph{size tables}. Each entry of the size table is the accumulated count of values in the corresponding subtree. A node contains a value if a table entry is bigger than the search key corresponding to it. When a balanced node is encountered, the search process falls back to the original radix search algorithm \Cref{sec:rb-tree-radix-search}. 

\todo{Update RrbTree-Find-Index to use binary search}
\begin{listing}[ht!]        
    \caption{Pseudocode for the relaxed radix search implementation}
    \label{lst:rrb-tree-relaxed-radix-search}        

    \begin{algorithmic}        
        \Function{RrbTree-Find-Index}{sizes, idx}
            \State candidate \la 0

            \If{candidate < \m\ - 1 \And\ sizes[candidate] <= idx}
                \State candidate++
            \EndIf

            \State \Return candidate
        \EndFunction

        \State

        \Function{RrbTree-Relaxed-Radix-Search}{root, key}
            \State node \la\ root
            \State idx \la\ key
            
            \For{level \la\ root\ts{height} - 1, 1}
                \If{node\ts{sizes} = \nil{}}
                    \State index \la\ (key $\ggg$ (level * x)) \& mask
                    \State node \la\ node[index]
                \Else
                    \State sizes \la\ node\ts{sizes}
                    \State index \la\ RrbTree-Find-Index(sizes, idx)
                    \State node \la\ node[index]

                    \If{index != 0}
                        \State idx \la\ idx - sizes[index - 1]
                    \EndIf                    
                \EndIf            
            \EndFor

            \State index \la\ idx \& mask
            \State \Return {node[index]}
        \EndFunction
    \end{algorithmic}
\end{listing}
