\thispagestyle{plain}

\begin{center}
    \includegraphics[width=8cm, angle=0, trim=10 10 10 10, clip]{images/ferris-climbing.png}

    \phantomsection
    \addcontentsline{toc}{section}{Reading notes}

    \section*{Reading notes}
    \begin{justify}
        The links to the LaTeX source code and the latest version of this document can be found at \url{https://abishov.com/thesis/}. The implementation, documentation, and visualization demo can be found at \url{https://abishov.com/pvec-rs}.

        If you notice any typos while reading the document, or have any feedback in general, feel free to open an issue at \url{https://github.com/arazabishov/thesis/issues} or send me an email at \href{mailto:araz@abishov.com}{\nolinkurl{araz@abishov.com}}.

        \paragraph{Colophon}
        The illustration above with Ferris\footnote{Unofficial mascot of Rust: \url{https://rustacean.net/}} sitting on top of an \treerrb{} was kindly prepared by Vanessa Tesorone. The design of the reading notes and the idea to use Rust's mascot for the document decoration was inspired by the master's thesis of Erik Vesteraas\footnote{\url{http://erik.vestera.as/thesis/}}.
    \end{justify}

    \subsection*{Typographic conventions}
    \begin{tabular}{ r l }
        Clickable link & \href{https://www.rust-lang.org/}{Rust Programming Language} \\
        Inline code and types & \mintinline{rust}{Vec::new()} \\
        Project or library name & \pvecrs{} \\
    \end{tabular}

\end{center}
