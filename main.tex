\documentclass[12pt,a4paper,english,twoside,openright]{report}
\usepackage{parskip}

\usepackage[includeheadfoot, inner=3.5cm, outer=3.5cm]{geometry}
\setlength{\headheight}{15pt}

\usepackage{emptypage}
\usepackage{pdfpages}

\usepackage{fancyhdr}
\pagestyle{fancy}
\fancyhf{}
\fancyhead[LE,RO]{\thepage}
\renewcommand{\sectionmark}[1]{ \markright{\thesection\ #1}{} }
\fancyhead[RE,LO]{\rightmark}

\usepackage{ragged2e}
\usepackage{adjustbox}
\usepackage{float}

\usepackage{multirow, array}
\usepackage{arydshln}

\usepackage[utf8]{inputenc}
\usepackage[T1]{fontenc,url}
\urlstyle{sf}

\usepackage[binary-units=true]{siunitx}
\DeclareSIUnit{\nothing}{\relax}
\newcommand{\kilo}[1]{\SI{#1}{\kilo\nothing}}
\newcommand{\mega}[1]{\SI{#1}{\mega\nothing}}
\newcommand{\micros}[1]{\SI{#1}{\micro\second}}
\newcommand{\millis}[1]{\SI{#1}{\milli\second}}
\newcommand{\seconds}[1]{\SI{#1}{\second}}
\newcommand{\mbytes}[1]{\SI{#1}{\mega\byte}}
\newcommand{\gbytes}[1]{\SI{#1}{\giga\byte}}

\usepackage{amssymb}
\usepackage{tikz}
\usetikzlibrary{matrix,arrows,backgrounds,fit,positioning}

\usepackage{forest}
\usepackage{pgfplots}
\usepgfplotslibrary{units}
\usepgfplotslibrary{groupplots}
\pgfplotsset{compat=1.16}

\usepackage{minted}
\newmintinline[ir]{rust}{}
\newmintedfile[rustfile]{rust}{}

\usepackage[noend]{algorithm,algpseudocode}
\makeatletter
\renewcommand{\ALG@beginalgorithmic}{\small}
\makeatother

\usepackage{babel,textcomp,csquotes,varioref,graphicx}

\usepackage[
  backend=biber,
  maxbibnames=6,
  isbn=false,
  doi=false,
  defernumbers=true
]{biblatex}
\addbibresource{refs.bib}
\setlength\bibitemsep{\baselineskip}

\usepackage[
  pdftitle={An RRB-tree based vector for Rust},
  pdfauthor={Araz Abishov},
  citecolor=linkblue,
  urlcolor=linkblue,
  colorlinks=true,
  linkcolor=black,
  unicode=true
]{hyperref}
\urlstyle{same}
\usepackage{cleveref}
\usepackage{xurl}

\raggedbottom

\newcommand{\dir}[1]{\mintinline[breaklines]{bash}{#1}}
\newcommand{\rust}[1]{\mintinline[breaklines]{rust}{#1}}
\newcommand{\type}[1]{\rust{#1}}
\newcommand{\crate}[1]{\emph{\textbf{#1}}}

\newcommand{\bigo}[1]{$\mathcal{O}(#1)$}
\newcommand{\bigochar}[1]{$\mathcal{O}$}
\newcommand{\range}[1]{#1}

\newcommand{\pvecrs}{\crate{pvec-rs}}
\newcommand{\imrs}{\crate{imrs}}
\newcommand{\rpds}{\crate{rpds}}
\newcommand{\rayon}{\crate{rayon}}

% Used to refer to the RRB-tree type
% provided by pvec-rs.
\newcommand{\rrbtree}{\type{RrbTree}}

% This macro is used to refer to the RRB-tree
% as a concept, rather than a concrete type
% provided by pvec-rs.
\newcommand{\treerrb}{RRB-tree}
\newcommand{\treerb}{RB-tree}

\newcommand{\refcell}{\type{RefCell}}
\newcommand{\boxptr}{\type{Box}}
\newcommand{\arc}{\type{Arc}}
\newcommand{\rc}{\type{Rc}}

\newcommand{\rbtree}{\type{RbTree}}
\newcommand{\rbvec}{\type{RbVec}}
\newcommand{\rrbvec}{\type{RrbVec}}
\newcommand{\pvec}{\type{PVec}}
\newcommand{\imrsvec}{\type{ImVec}}
\newcommand{\stdvec}{\type{Vec}}

\newcommand{\m}{$m$}
\newcommand{\h}{$h$}
\newcommand{\x}{$x$}
\newcommand{\nil}{\emph{NIL}}
\newcommand{\la}{$\leftarrow$}
\newcommand{\ts}[1]{\textsubscript{#1}}
\newcommand{\n}{$N$}

\algnewcommand\And{\textbf{and}}
\algnewcommand\In{\textbf{in}}

\definecolor{linkblue}{HTML}{008CFF}
\definecolor{morange}{HTML}{FF9800}
\definecolor{mpurple}{HTML}{512DA8}
\definecolor{mred}{HTML}{D32F2F}
\definecolor{mgreen}{HTML}{689F38}

\title{A Vector Implementation Based On RRB-Tree for Rust}
\author{Araz Abishov}

\begin{document}

\pagenumbering{roman}
\includepdf[pages={-}]{forside/forside.pdf}
\setcounter{page}{5}

\pdfbookmark[0]{Front matter}{bm-frontmatter}
\pdfbookmark[1]{Contents}{bm-toc}
\tableofcontents{}

\cleardoublepage
\vspace*{2cm}
\thispagestyle{plain}

\begin{center}

\phantomsection
\addcontentsline{toc}{section}{Abstract}

\section*{Abstract}

\end{center}

\todo{Abstract}

\section*{Structure}
\begin{itemize}
    \item Abstract should be connected to title
    \item Often consists of 3 stages
    \begin{itemize}
        \item 2 stages (why?) are general (shade the light on work, motivation)
        \item 1 stage (what?) should be more specific about this paper (mention what has been achieved)
    \end{itemize}
\end{itemize}

\section*{Thoughts}
\begin{itemize}
    \item Influence on software design
    \item Immutability, concurrency and functional programming languages
    \item Applications in particular problem domains
    \item Why vectors in particular?
\end{itemize}


\cleardoublepage
\vspace*{2cm}
\thispagestyle{plain}

\begin{center}

\phantomsection
\addcontentsline{toc}{section}{Acknowledgements}

\section*{Acknowledgements}

\end{center}

I would like to thank my supervisors, Martin Steffen and Volker Stolz, for their guidance and feedback. Without their patience and support, as well as trust in my own idea, this venture would not have been possible.

My deepest appreciation goes to my family and my friends for their moral support, their continuous encouragement, and their help whenever it was needed.

Finally, I would like to thank the Rust community for being so welcoming and supportive. In particular, I am grateful to Nicholas Matsakis for inspiring me to work on this project.


\cleardoublepage
\vspace*{2cm}
\thispagestyle{plain}

\begin{center}

\phantomsection
\addcontentsline{toc}{section}{Reading notes}

\section*{Reading notes}

\end{center}

\todo{Reading notes}


\cleardoublepage

\pagenumbering{arabic}
% The experience of dealing with these errors often can be frustrating, especially for programmers new to Rust.

% However, the Rust developers have spent a large amount of time working to improve the error messages to ensure that they are clear and actionable. Don’t let your eyes gloss over while reading Rust errors!

\chapter{Introduction}
% Total volume of the introduction should not exceed 5-6 paragraphs

% 1. Why Rust is cool and worth your time. Why: a, b, c
Rust is a modern, open-source programming language with a focus on memory safety and performance. Its rich type system not only eliminates several classes of bugs but also makes the language powerful and expressive for building high-level programs such as web servers and command-line interface applications. With direct access to computer memory and hardware, Rust is a great language for embedded and low-level programming as well.

% 2. What are the challenges of working with Rust? a, b, c
However, due to the emphasis on memory safety and rigorous checks enforced by the compiler, it is common to get errors when compiling a Rust program. One of the rules is the forbidden simultaneous sharing and mutation of an object. Often, errors caused by this rule can be avoided by better design, but sometimes, the best resolution is to clone the value before sharing it. Naive cloning by copying, however, is an expensive operation both in terms of time and space. Thus, resorting to it as a solution, especially for large-sized collections might be in-efficient. 

% 3. Persistent data structures can help: why, and how? a, b, c
Persistent data structures are the data structures that provide access to all their previous versions. Often persistence is achieved through copying, and thus, various data structure designs have been developed throughout the years to optimize for this operation. The standard library of the Scala programming language\footnote{\url{https://www.scala-lang.org/}} provides a persistent vector implementation that is efficient not only when copied but also across all operations.

% 4. Okay, now we understand what are the challenges, and why Rust and persistent 
%      data structures are a match made in heaven. What is my role in this story?
This thesis presents \pvecrs{}, a project that contributes a vector implementation with efficient clone operation that borrows ideas from persistent data structures. The project explores novel approaches to optimize vector’s performance by leveraging type system of Rust, as well as aiming to achieve convenient, idiomatic interface familiar to developers. The proposed optimizations are evaluated and discussed based on results of the sequential and parallel tests.

% 5. Let's learn a little bit more about Rust and Persistent data structures first
In this introductory chapter, we will look at the background for the \pvecrs{} project. First, in section \ref{sec:rust}, we will look at the Rust programming language. Then, in section \ref{sec:psds}, we will take a closer look at persistent data structures and their categories. Finally, section \ref{sec:contributions} gives an overview of the contributions made in this project.

In chapter 2, the data structure that serves as the foundation for the persistent vector in Rust -- the RRB-Tree, is presented. In chapter 3, the \pvecrs{} project, its interface, and optimizations specific to Rust are discussed in detail. Chapter 3 focuses on the performance evaluation of the data structures and discusses ways in which it could be extended and improved.

% Take a look at the paper about fork/join
% Stack, Heap, and garbage collection

\section{The Rust programming language}
\label{sec:rust}

\subsection{Ownership and borrowing}
\todo{Ownership and borrowing}

\subsection{Reference counting and memory management}
\todo{Reference counting and memory management}

\section{Persistent data structures}
\label{sec:psds}

A vast majority of modern programming languages are equipped with a standard library --- a set of constructs and utilities aimed to improve the developer's productivity. 

A significant part of standard library consists of commonly used data structures such as lists, sets, and maps, which often provide operations for reading and writing data. 

Modifying or \emph{mutating} an \emph{ephemeral} data structure implies that we no longer will have access to its older version. In contrast, a \emph{persistent} data structure allows access to any version, old or new, at any time \cite{making-data-structures-persistent}. 

Persistent data structures could be classified based on the operations which they offer over their versions:
\begin{itemize}
    \item \textit{Partial persistence} --- In this persistence model, we can query any previous version of the data structure, but we can update only the newest version. The versions are linearly ordered. 
    \item \textit{Full persistence} --- Both access and updates are allowed on all versions. The versions can be visualized as a branching tree.
    \item \textit{Confluent persistence} --- In addition to the previous operations, it offers combination operation to merge more than one previous version to output a new single version. The versions form a directed acyclic graph \cite{fully-persistent-lists-with-catenation}.  
    \item \textit{Functional persistence} --- This model takes its name from functional programming where objects are immutable. In comparison to the previous models, it prohibits change of the internal representation of the data structure \cite{purely-functional-data-structures}. 
\end{itemize}

While persistence can be achieved by simple copying, the performance of a modification operation quickly becomes unacceptable. This led to research and development of more efficient solutions, which were often designed to solve particular problems. A lack of general purpose collections, which guarantee uniformly good performance across different operations, results in challenges with software development. 

A persistent \emph{vector}, also known as a one-dimensional growable array, is very inefficient if implemented naively. Each update operation will cause a full copy of the underlying array, consuming additional amount of memory and processor time. 

However most operations modify only some parts of data structures, a complete copy is redundant. A better approach is to exploit the similarity between the new and old versions by \emph{sharing} structure between them. For example, instead of representing a data structure as a single block of memory, it could be broken down into smaller pieces or \emph{nodes}, which are linked together in the form of a \emph{tree}. Since modifications only apply to some nodes, the rest of them still remain unchanged and can be shared without copying. 

The first persistent vector implementation which offered good performance across a broad range of operations was introduced by Rich Hickey in the Clojure programming language. It was based on Phil Bagwell's Hash Array Mapped Trie \cite{ideal-hash-trees}, which offers practically constant runtime for push, update, and access operations, and it is a \emph{fully} persistent data structure. 

Later Phill Bagwell introduced a confluently persistent vector based on Relaxed Radix Balanced Tree \cite{efficient-immutable-vectors}, which has improved performance of the concatenation operation significantly and became a foundation for Scala's standard library vector implementation. 

The project presented in the thesis contributes an efficient persistent vector implementation for Rust based on Relaxed Radix Balanced Tree, which takes advantage of the Rust's type system to improve performance, guarantee thread safety and provide an idiomatic application programming interface. In order to ensure the best possible average performance, its internal representation switches from standard vector to RRB-Tree during runtime. 

\section{Contributions}
\label{sec:contributions}
\todo{Contributions}

% TODO: give a definition to core (discrete operations)
%   This is how you can distinguish core from confluent operations:
%     * Core operations do not produce relaxed nodes in \rrbtree{} unless the nodes on the path to the leaves are already relaxed
%     * Alternative name for core is discrete

% TODO: define what is transience: https://clojure.org/reference/transients
\chapter{Background}

\section{Radix balanced tree}
\label{chapter:radix-balanced-tree}

Radix balanced tree or \treerb{} is an \m{}-ary tree that uses integers as keys to find values. The data structure was pioneered by Rich Hickey as the foundation for the persistent vector implementation in Clojure \cite{the-clojure-programming-language}.

\treerb{} consists of nodes which reference other subtrees or values. We will be referring to the former and latter types of nodes as \emph{branches} and \emph{leaves} correspondingly. The number of subtrees and values in the node is configurable and will be defined as \m{} or the \emph{branching factor}. The branching factor can be any number that is a power of 2, allowing an efficient radix search implementation.

The higher the value of \m{} is, the wider and shallower the tree will be. From now on the height of a tree will be referred to as \h{}, where $h_{max}$ is the upper boundary:

\begin{equation} \label{eq:h-max}
    h_{max} = log_m(n)
\end{equation}

When \m{} is a large value, for example, 32, the tree becomes shallow, and the complexity of accessing values by traversing the tree from the root to a leaf becomes \emph{effectively} constant. For example, if the maximum value of 32 bit signed integer is substituted for $n$ in \Cref{eq:h-max} then $h_{max}$ will never exceed 7 levels. Thus, due to its strong performance guarantees, \treerb{} serves as a solid foundation for a general-purpose persistent vector implementation.

In the following sections, we will take a look at \treerb{} algorithms used to implement vector operations\footnote{For a formal definition of \treerb{}, refer to \cite{improving-performance-through-transience}}.

\subsection{Radix search}
\label{sec:rb-tree-radix-search}

Before trying to understand how operations such as push, pop, and update work, let's take a look at how to access values in \rbtree{}. The lookup mechanism is called \emph{radix search}, a fundamental algorithm which forms the basis for other operations.

Conceptually, the idea behind the search in an ordered tree-like structure boils down to picking correct nodes based on the given key. If there is a value corresponding to the key, we stop searching and return the value. Otherwise, an empty value or error is returned.

The lookup mechanism depends on organization of nodes in the tree. Let's consider an example where each node has at most two children, called \emph{binary tree}. A binary tree where every node fits a specific ordering property\footnote{For every node $n$, all left descendents must be less than or equal than $n$, and all right descendents must be strictly bigger than $n$.} is called \emph{binary search tree}. While searching, the key at each node is compared to the search key. All that is important is whether the key in the node is less than, equal to, or greater than the search key. The search is continued until an exact match or reaching a leaf node.

Another tree-like data structure known as trie, is interesting because its nodes do not store complete keys. Instead, each node stores only part of it. Since tries are often used for the prefix search of words, their nodes store characters. Each path down the tree may represent a word. A node in a trie could have anywhere from 0 through the size of the alphabet children. For example, the english alphabet has 26 letters, meaning that each node might have up to 26 children.

The lookup procedure for tries involves breaking down the search key into multiple subkeys, which are used to pick corresponding sub-tries. For example, the key "car" will be broken down into several smaller keys such as "c", "a", "r". If there is a value present at the last node of the path, it is returned. Otherwise, an empty value or error is returned.

\subsubsection*{Bit partitioning}
Radix search accepts a key of integer type as an argument. It can be viewed as a composite key, where each subkey is represented by a sequence of bits. The idea is to divide the key into blocks of bits, where each block forms an index specific to the tree node. The count of bits in each of those chunks can be derived from the branching factor and will be named as \emph{bits per level}.

The \emph{bits per level} or \x{} is the count of bits used to address \m{} nodes in the binary numeral system:

\begin{equation}
    \label{eq:bits-per-level}
    x = log_2(m).
\end{equation}

For example, with \m{} equal to 16, the maximum index value is 15. 15 converted to the binary form is $1111_2$, which evidently requires 4 bits of space. If we substitute \m{} into \Cref{eq:bits-per-level}, we will get the same value.

\subsubsection*{Extracting subkeys}
Now when the size of the subkey is known, the next step is to identify its location. The count of subkeys within the search key depends on \h. Each new level in \rbtree{} will use \x{} additional bits of space. For instance, a search key addressing an element of the tree of ${h = 3}$ and ${x = 2}$ will consist of 3 subkeys taking up 6 bits of space in total.

Subkeys are arranged in the order from the most to the least significant bits, where the most significant block of bits is a key used to access child node of the root. Each following key is used to find a child node on the corresponding tree level.

Knowing the depth at which a node is located and the count of bits per level, the value of the key can be calculated using bitwise operations such as logical shift and masking. Let's take a look at a mechanism used to extract the correct subkey.

In the following example there is a byte which represents a key equal to 54. Let's assume that it belongs to the tree where \m{} is 4, \x{} is 2 and \h\ is 3.

\begin{equation}
    54_{10} = 00110110_2
\end{equation}

Since there are three levels, we have only three subkeys: $11_2$, $01_2$ and $10_2$. Let's say that we are interested in extracting subkey for the child node on the second level or $01_2$.

First, let's get rid of the bits following the subkey of our interest. Logical right shift operation\footnote{A logical right shift is a bitwise operation that shifts all the bits of its first operand to the right by number of bits specified in the second operand.} denoted as $\ggg$, will push specified count of 0s into key $k$, where $l$ is the \emph{level} at which current node is located:

\begin{equation}
    k \ggg ((l - 1) * x).
\end{equation}

The key $k$ will be shifted by 2 bits, because the node from the example is located at $l = 2$ in the tree of $x = 2$. The result of operation is $00001101_2$. As the result, the "tail" of the key is truncated.

The next step is to get rid of bits preceding the subkey by masking them to 0. An operator used for this is known as bitwise "and" and it will be applied to the result of shifting operation.

A bitwise "and" takes two equal-length binary representations and executes the "and" operation for each pair of the corresponding bits. If both bits in the compared position are 1, the bit in the resulting binary representation is 1; otherwise, the result is 0. Here is an example, where the first and the second operands are the key and mask respectively:

\begin{equation}
    00001101 \ \& \ 00000011 = 00000001.
\end{equation}

Only the last two bits of the mask are set to 1, which means that all bits of the key except last two will be masked to 0. The result of the "and"-ing operation will be the value of the subkey.

A mask must be of the same type as a key and can be calculated from the branching factor \m:

\begin{equation}
    \mathit{mask} = m - 1.
\end{equation}

If \m{} is equal to 4, the maximum subkey value will be 3, which equals to 00000011\ts{2}.

\begin{figure}
    \caption{Visualization of the radix search algorithm.}
    \label{fig:rb-tree-example-1}

    % TODO: font=\ttfamily
    \centering
    \begin{tikzpicture} [
        node/.style = {
            matrix of nodes,
            nodes = { draw, minimum width = 6mm, minimum height = 8mm, anchor = center},
            font = \small,
            nodes in empty cells
        },
        value/.style = {
            matrix of nodes,
            nodes = { draw = none, minimum width = 4mm, minimum height = 4mm, anchor = center, rotate = 90 },
            font = \small,
            nodes in empty cells
        },
        edge/.style = { ->, shorten >= 4pt }
    ]
        \node[] (index) at (current page.north west) { $104_{10}$ = $01 10 10 00_{2}$ };

        \matrix[node] (node-1-1) [below right = 8mm and 1cm of index] { 00 & 01 & 10 & 11 \\ };

        \scoped[on background layer] {
            \node[fit=(node-1-1-1-1), fill=color-node, inner sep = 0pt]   {};
            \node[fit=(node-1-1-1-2), fill=color-path, inner sep = 0pt]   {};
            \node[fit=(node-1-1-1-3), fill=color-node, inner sep = 0pt]   {};
            \node[fit=(node-1-1-1-4), fill=color-node, inner sep = 0pt]   {};
        }

        \matrix[node, inner sep = 0pt] (node-2-2) [below left = 8mm and 1mm of node-1-1.south] { 00 & 01 & 10 & 11 \\ };
        \matrix[node, fill = color-node, inner sep = 0pt] (node-2-3) [below right = 8mm and 1mm of node-1-1.south] { & & & \\ };
        \matrix[node, fill = color-node, inner sep = 0pt] (node-2-1) [left = 2mm of node-2-2.west] { & & & \\ };
        \matrix[node, fill = color-node, inner sep = 0pt] (node-2-4) [right = 2mm of node-2-3.east] { & & & \\ };

        \scoped[on background layer] {
            \node[fit=(node-2-2-1-1), fill=color-node, inner sep = 0pt]   {};
            \node[fit=(node-2-2-1-2), fill=color-node, inner sep = 0pt]   {};
            \node[fit=(node-2-2-1-3), fill=color-path, inner sep = 0pt]   {};
            \node[fit=(node-2-2-1-4), fill=color-node, inner sep = 0pt]   {};
        }

        \draw[edge, out=225, in=45] (node-1-1-1-1.south) to (node-2-1.north);
        \draw[edge, out=225, in=45] (node-1-1-1-2.south) to (node-2-2.north);
        \draw[edge, out=315, in=135] (node-1-1-1-3.south) to (node-2-3.north);
        \draw[edge, out=315, in=135] (node-1-1-1-4.south) to (node-2-4.north);

        \matrix[node, fill = color-node, inner sep = 0pt] (node-3-2) [below left = 8mm and 1mm of node-2-2.south] { & & & \\ };
        \matrix[node, inner sep = 0pt] (node-3-3) [below right = 8mm and 1mm of node-2-2.south] { 00 & 01 & 10 & 11 \\ };
        \matrix[node, fill = color-node, inner sep = 0pt] (node-3-1) [left = 2mm of node-3-2.west] { & & & \\ };
        \matrix[node, fill = color-node, inner sep = 0pt] (node-3-4) [right = 2mm of node-3-3.east] { & & & \\ };

        \scoped[on background layer] {
            \node[fit=(node-3-3-1-1), fill=color-node, inner sep = 0pt]   {};
            \node[fit=(node-3-3-1-2), fill=color-node, inner sep = 0pt]   {};
            \node[fit=(node-3-3-1-3), fill=color-node, inner sep = 0pt]   {};
            \node[fit=(node-3-3-1-4), fill=color-path, inner sep = 0pt]   {};
        }

        \draw[edge, out=225, in=45] (node-2-2-1-1.south) to (node-3-1.north);
        \draw[edge, out=225, in=45] (node-2-2-1-2.south) to (node-3-2.north);
        \draw[edge, out=315, in=135] (node-2-2-1-3.south) to (node-3-3.north);
        \draw[edge, out=315, in=135] (node-2-2-1-4.south) to (node-3-4.north);

        \matrix[node, fill = color-node, inner sep = 0pt] (node-4-2) [below left = 8mm and 1mm of node-3-3.south] { & & & \\ };
        \matrix[node, fill = color-node, inner sep = 0pt] (node-4-3) [below right = 8mm and 1mm of node-3-3.south] { & & & \\ };
        \matrix[node, fill = color-node, inner sep = 0pt] (node-4-1) [left = 2mm of node-4-2.west] { & & & \\ };
        \matrix[node, inner sep = 0pt] (node-4-4) [right = 2mm of node-4-3.east] { 00 & 01 & 10 & 11 \\ };

        \scoped[on background layer] {
            \node[fit=(node-4-4-1-1), fill=color-path, inner sep = 0pt]   {};
            \node[fit=(node-4-4-1-2), fill=color-node, inner sep = 0pt]   {};
            \node[fit=(node-4-4-1-3), fill=color-node, inner sep = 0pt]   {};
            \node[fit=(node-4-4-1-4), fill=color-node, inner sep = 0pt]   {};
        }

        \draw[edge, out=225, in=45] (node-3-3-1-1.south) to (node-4-1.north);
        \draw[edge, out=225, in=45] (node-3-3-1-2.south) to (node-4-2.north);
        \draw[edge, out=315, in=135] (node-3-3-1-3.south) to (node-4-3.north);
        \draw[edge, out=315, in=135] (node-3-3-1-4.south) to (node-4-4.north);

        \matrix[value] (node-5-1) [below = 0mm of node-4-1.south] { 139 & 140 & 141 & 142 \\ };
        \matrix[value] (node-5-2) [below = 0mm of node-4-2.south] { 143 & 144 & 145 & 146 \\ };
        \matrix[value] (node-5-3) [below = 0mm of node-4-3.south] { 147 & 148 & 149 & 150 \\ };
        \matrix[value] (node-5-4) [below = 0mm of node-4-4.south] { 151 & 152 & 153 & 154 \\ };
    \end{tikzpicture}
\end{figure}

Let's put everything together and review the radix search step by step based on the concrete example. In \Cref{fig:rb-tree-example-1}, we can see a part of the \rbtree{} which represents elements of persistent vector in the range [92, 107].

The branching factor of the tree is equal to 4, which means that 2 bits will be used for the subkey representation. The mask is equal to 3 or 00000011 in the binary representation.

The index is of the \emph{unsigned byte} type, which has a capacity\footnote{The capacity of any integer type can be calculated by taking the base of the binary system to the power of bits available for its representation. For unsigned byte, it will be $2^8 = 256$.} of 256. In practice, the index type is usually a 32 or a 64 bit integer, which has significantly bigger capacity.

The height of the tree in \Cref{fig:rb-tree-example-1} is 4. The goal is to lookup the contents of an element corresponding to the key 104, which is equal to 01101000\ts{2}. \Cref{lst:rb-tree-radix-search} outlines the radix search algorithm implementation.

The search process starts by initializing the node and level variables to root and $height - 1$ correspondingly. The for loop runs until the leaf node level is reached, where each step towards it involves selecting a child node by using bit shifting and masking. After exiting the for loop, function returns the value from the leaf node.

\begin{listing}[ht!]
    \caption{Pseudocode for RB-Tree's radix search implementation.}
    \label{lst:rb-tree-radix-search}

    \begin{algorithmic}
        \Function{RbTree-Radix-Search}{root, key}
            \State node \la\ root

            \For{level \la\ root\ts{height} - 1, 1}
                \State index \la\ (key $\ggg$ (level * x)) \& mask
                \State node \la\ node[index]
            \EndFor

            \State index \la\ key \& mask
            \State \Return {node[index]}
        \EndFunction
    \end{algorithmic}
\end{listing}

\subsection{Update}
A function for replacing values in \rbtree{} is called \emph{update}. Emphemeral update is \emph{destructive}, meaning that the original \emph{version} of the data structure will be no longer available after the operation is executed. For \emph{persistent data structures}, update results in the new \emph{version} including the updated value without the original becoming unavailable.

Version is something that differentiates one instance of the \rbtree{} from another. Instead of using explicit versioning such as assigning unique identifiers to instances of \rbtree{}, we will rely on a memory address as a unique identifier.

\subsubsection*{Path copying}
% PATH COPYING AND COPY-ON-WRITE behavior

% Logically each node is a copy-on-write array that contains subtrees or elements.
% To minimize copying while retaining full persistence, we perform path copying: We copy all nodes on the path down to the value we’re about to update or insert and replace the value with the new one when we’re at the bottom.

Radix search is used to find the value corresponding to the given key. In order to ensure that the original version of \rbtree\ stays unmodified, each node on the way down from the root to the value will be copied. This process is also know as \emph{path copying} \cite{planar-point-location}. It is important to emphasize that nodes which are not a part of the path are reused, instead of being recreated.

In order to get a sense of how well path copying performs, let’s review the whole process in reverse. The leaf node update creates a copy of underlying array of size \m{}. Then, the \m{} sized parent branch node is copied as well, where the fresh copy of the leaf is used instead of the original. The other children of the parent are reused. The same process applies to the grand parent and so on all the way up to the root.

To summarize, the update operation performs the \h{} count copies of \m{} sized nodes, where \h{} is the maximum height of the tree. As described in \Cref{eq:h-max}, \h{} is bound by \bigo{log_m(n)}, which results in {\bigo{m * log_m(n)}} complexity of the update operation.

For large branching factors \m\ such as 32, the performance becomes \bigo{1} in practice. For example, if the tree is completely full, where the count of all elements $n$ is bound by the maximum value of 32 bit integer\footnote{2147483647}, \h{} will be at most 7.

\begin{listing}[ht!]
    \caption{Pseudocode for RB-Tree's update implementation.}
    \label{lst:rb-tree-update}

    \begin{algorithmic}
        \Function{RbTree-Update}{root, key, value}
            \State newRoot \la\ clone(root)
            \State node \la\ newRoot

            \For{level \la\ root\ts{h} - 1, 1}
                \State index \la\ (key $\ggg$ (level * x)) \& mask
                \State newChildNode \la\ Clone(node[index])
                \State node[index] \la\ newChildNode
                \State node \la\ newChildNode
            \EndFor

            \State index \la\ key \& mask
            \State node[index] \la\ value
            \State \Return newRoot
        \EndFunction
    \end{algorithmic}
\end{listing}

From the implementation standpoint, the main difference between update and radix search is that every visited node must be copied, including root. The return value is the root node for the new version of the tree.


\subsection{Push}
% This are not references, but values themselves
References to values are stored at the leaf nodes of the tree. If the number of values stored in the structure is less than branching factor, then the root node itself will be a leaf. Otherwise, capacity of the tree is increased by adding intermediate branch nodes.


The push operation is used to add new elements to the \rbtree{}. It employs the \emph{path copying} mechanism to minimize the cost and maximize performance of the operation. It accepts a root node and a new value as arguments, and returns a new version of the tree.

A key for the new element is equal to the count of elements in \rbtree{}, also known as the \emph{size} of the tree. For example, if the size is 9, the key for the new value will also be 9. The size will be incremented after the operation is executed.

The main difference between the update and the push operation is that the latter one has to take care of allocating more space for new elements. If the rightmost leaf node has available slots, then behavior of push is very similar to update. Otherwise, there are two additional scenarios in which either the rightmost leaf is completely full or there is no more space left in the root node.

The solution for the first case comes from radix search. Given a unique key, radix search calculates a path that never leads to a full leaf node. What might happen is that a path will include nodes which are not created yet. This can be solved by generating them on demand.

The second case, also known as \emph{root overflow}, occurs when the \emph{size} of the tree exceeds its \emph{capacity}. Capacity is the maximum number of elements a tree can accommodate, which can be calculated based on the height \h{} and the branching factor \m{} of \rbtree{}:

\begin{equation}
	c = m^h.
\end{equation}

The implementation of this calculation is based on the left bit shift operation, described in \Cref{lst:rb-tree-capacity}.

The Root overflow can be solved by adding a new level to the tree. This is done by creating a new root node and setting the old root as the first child of the new root. The rest is handled by generating new nodes in the path to the new value. For more details, see \Cref{lst:rb-tree-push}.

\begin{listing}[ht!]
    \caption{Pseudocode for RB-Tree's capacity implementation.}
    \label{lst:rb-tree-capacity}

    \begin{algorithmic}
        \Function{RbTree-Capacity}{height}
		\State \Return m $\ll$ ((height - 1) * x)
        \EndFunction
    \end{algorithmic}
\end{listing}

\begin{listing}[ht!]
    \caption{Pseudocode for RB-Tree's push implementation}
    \label{lst:rb-tree-push}

    \begin{algorithmic}
        \Function{Rb-Tree-Push}{root, value}
            \State newRoot \la\ \nil{}

            \If{RbTree-Capacity(root\ts{height}) <= root\ts{size}}
                \State newRoot \la\ Create-Node()
                \State newRoot[0] \la\ Clone(root)
                \State newRoot\ts{height} \la\ root\ts{height} + 1
            \Else
                \State newRoot \la\ Clone(root)
            \EndIf

            \State node \la\ newRoot
            \State key \la\ newRoot\ts{size}

            \For{level \la\ newRoot\ts{height} - 1, 1}
                \State index \la\ (key $\ggg$ (level * x)) \& mask

                \State childNode \la\ node[index]
                \State newChildNode \la\ \nil{}

                \If{childNode = \nil{}}
                    \State newChildNode \la\ Create-Node()
                \Else
                    \State newChildNode \la\ Clone(childNode)
                \EndIf

                \State node[index] \la\ newChildNode
                \State node \la\ newChildNode
            \EndFor

            \State index \la\ key \& mask
            \State node[index] \la\ value

            \State newRoot\ts{size} \la\ newRoot\ts{size} + 1
            \State \Return newRoot
        \EndFunction
    \end{algorithmic}
\end{listing}

\subsection{Pop}
Pop is used to remove items from the \rbtree. Together with push, it conforms to the LIFO\footnote{Last In First Out} principle which is typical for the stack abstract data type. It accepts root as input and returns both a popped value and a new version of a structure.

Pop is responsible for reducing capacity of the \rbtree{}, including removal of branch and leaf nodes when they become empty. As with other operations, modifications are kept to the minimum by taking advantage of path copying algorithm.

\begin{listing}[ht!]
    \caption{Pseudocode for RB-Tree's pop implementation}
    \label{lst:rb-tree-pop}

    \begin{algorithmic}
        \Function{RbTree-Pop-Node}{node, key}
            \State newNode \la\ Clone(node)
            \State value \la\ \nil{}

            \If{node\ts{height} = 0}
                \State index \la\ key \& mask
                \State value \la\ newNode[index]
                \State newNode[index] \la\ \nil{}
            \Else
                \State index \la\ (key $\ggg$ (newNode\ts{height} * x)) \& mask
                \State value, childNode \la\ RbTree-Pop-Node(newNode[index], key)
                \State newNode[index] \la\ childNode
            \EndIf

            \If{newNode[0] = \nil{}}
                \State \Return value, \nil{}
            \Else
                \State \Return value, newNode
            \EndIf
        \EndFunction
        \State
        \Function{Pop}{root}
            \State value, newRoot \la\ RbTree-Pop-Node(root, root\ts{size} - 1)

            \If{newRoot[1] = \nil{})}
                \State newRoot \la\ newRoot[0]
            \EndIf

            \State \Return value, newRoot
        \EndFunction
    \end{algorithmic}
\end{listing}

Since the \rbtree{} is a complete\footnote{A complete n-ary tree is an n-ary tree which is maximally space efficient. It must be completely filled on every level except for the last level. However, if the last level is not complete, then all nodes of the tree must be "as far left as possible".} tree, all entries to the right of \nil{} must be absent as well. As shown in \Cref{lst:rb-tree-pop}, checking if the first entry is \nil{} is sufficient to understand if a node must be removed. If it is true, the empty child will be replaced with a \nil{} reference in the parent node.

A root is considered to be redundant if it contains only a single child node, which is true if the second entry is \nil{}.  The original root is demoted by replacing it with the first child node, which becomes the new root.

\subsection{Tail optimization for persistent vector}
\label{sec:tail-optimization}

In practice, changes are often applied to the end or \emph{tail} of the data structure. Stack is specifically optimized for such use cases, by offering the constant performance for the push and pop operations. Even though \rbtree\ has similar performance characteristics, its push and pop implementations include pesky constant factors in the form of \emph{radix search} and \emph{path copying} algorithms.

The \emph{tail} optimization is intended to offset this cost by reducing the count of the \rbtree\ accesses. Instead of adding or removing elements one by one, changes are batched in the array of size \m. This array could be thought of as a leaf node, which will be attached to the tree only when it is full.

\subsubsection*{Optimizing the push operation}
As shown in \Cref{lst:pvec-push}, the value is set into a cloned tail at the position tail\ts{size}. Since \rbtree\ is a complete tree, the rightmost leaf node is the only node which can be empty or semi-full. As the tail is the rightmost leaf node, its size is as an index for the new value.

If after update the tail is full, it will be pushed into a tree and replaced with an empty tail.

\begin{listing}[ht!]
    \caption{Tail optimization for persistent vector’s push implementation.}
    \label{lst:pvec-push}

    \begin{algorithmic}
        \Function{Pvec-Push}{vec, value}
        \State newTail \la\ Clone(vec\ts{tail})
        \State newTail[tail\ts{size}] \la\ value
        \State newTail\ts{size} \la\ tail\ts{size} + 1
        \State newRoot \la\ vec\ts{root}

        \If{newTail\ts{size} = m}
            \State newRoot \la\ RbTree-Push(vec\ts{root}, newTail)
            \State newTail \la\ Create-Node()
        \EndIf

        \State \Return Create-Vec(newRoot, newTail)
        \EndFunction
    \end{algorithmic}
\end{listing}

\subsubsection*{Optimizing the pop operation}
Since the tail might contain elements, pop has to remove them first before touching the \rbtree. If the tail is empty, it will be replaced with the rightmost leaf of \rbtree. See \Cref{lst:pvec-pop} for more details.

\begin{listing}[ht!]
    \caption{Tail optimization for persistent vector’s pop implementation}
    \label{lst:pvec-pop}

    \begin{algorithmic}
        \Function{Pvec-Pop}{vec}

        \State newTail \la\ Clone(vec\ts{tail})
        \State newRoot \la\ \nil{}

        \State value \la\ newTail[newTail\ts{size} - 1]
        \State newTail\ts{size} \la\ newTail\ts{size} - 1

        \If{newTail\ts{size} = 0}
            \State newRoot, newTail \la\ RbTree-Pop(vec\ts{root})
        \Else
            \State newRoot \la\ vec\ts{root}
        \EndIf

        \State \Return Create-Vec(newRoot, newTail)
        \EndFunction
    \end{algorithmic}
\end{listing}

\subsubsection*{Adapting the update and radix search operations}
Changes for both update and radix search are very similar, with the difference that update has to ensure that the original version of vector stays unmodified.

The radix search implementation has to take into account that some of the values can be in the tail. A value is located within the tree if the key is less than the tree size. In this case the search process is delegated to \rbtree. Otherwise, the index for value in the tail is calculated by subtracting the tree size from the key.

\begin{listing}[ht!]
    \caption{Adapting the update operation to support tail}
    \label{lst:pvec-update}

    \begin{algorithmic}
        \Function{Pvec-Update}{vec, key, value}

        \State root \la\ vec\ts{root}
        \State tail \la\ vec\ts{tail}

        \If{key < root\ts{size}}
            \State newRoot \la\ RbTree-Update(root, key, value)
            \State \Return Create-Vec(newRoot, tail)
        \Else
            \State newTail \la\ Clone(tail)
            \State newTail[key - root\ts{size}] \la\ value
            \State \Return Create-Vec(root, newTail)
        \EndIf
        \EndFunction
    \end{algorithmic}
\end{listing}

\begin{listing}[ht!]
    \caption{Adapting the radix search operation to support tail}
    \label{lst:pvec-radix-search}

    \begin{algorithmic}
        \Function{Pvec-Radix-Search}{vec, key}

        \State root \la\ vec\ts{root}
        \State tail \la\ vec\ts{tail}

        \If{key < root\ts{size}}
            \State \Return RbTree-Radix-Search(root, key)
        \Else
            \State \Return tail[key - root\ts{size}]
        \EndIf
        \EndFunction
    \end{algorithmic}
\end{listing}

\section{Relaxed radix balanced tree}
% A couple of words here on what is RRB-Tree, what it offers, and how is it different compared to RB-Tree.
%  * A confluent persistent data structure.
%  * Relaxation as a way to achieve efficient concatenation and splitting.

% Then, say what this chapter is about:
%  * Which constraints are relaxed and which constraints are enforced:
%   * The relaxed child count and completeness of nodes
%   * How does RRB-Tree guarantee that height won't exceed certain limits
%  * The concatenation and splitting algorithms.
%  * Changing the following algorithms to accommodate relaxed constraints:
%   * Radix search

Relaxed radix balanced tree or \treerrb{}, extends \treerb{} to support concatenation and splitting in \bigo{log(n)} rather than linear time without compromising the performance of other operations.

An invariant that \treerb{} maintains is that all nodes except the right-most ones have to be full. This enables a simple radix search implementation, but on the other hand, makes efficient sub-linear concatenation impossible.

The \treerrb{} relaxes this constraint by allowing nodes to be partially full, and introduces a rebalancing algorithm to ensure that the tree height does not exceed the \bigo{log(n)} bound to keep performance guarantees for other operations.

In this section, we will take a look at how \treerrb{} works, specifically the concatenation algorithm and the relaxed variant of the radix search.

\subsection{Memory layout}
With relaxed \treerrb{} constraints, there is no reliable way to calculate the size of the subtree without additional metadata. Hence, the \emph{sizes} array is introduced to keep track of the subtree size. The \emph{sizes} array is \m{} elements long, where each value represents the subtree size at the corresponding index. The sizes table values are re-calculated when the subtree is modified, for example, when concatenating or pushing new values.

\subsection{Relaxed radix search}
When \treerb{} is relaxed, it is no longer possible to compute indices from a search key without additional metadata in the form of \emph{size} tables. Each entry of the size table is an accumulated number of values in the corresponding subtree.

When the radix search encounters a relaxed node, it compares the entire search key against entries in the size table. A subtree contains the desired value if the corresponding size table entry is bigger than or equal to the search key. Before descending into a subtree and repeating the search step, the search key is subtracted the size of the subtree.

When a balanced node is encountered, the search process falls back to the balanced version of the radix search algorithm \Cref{sec:rb-tree-radix-search}.

\begin{listing}[!ht]

    \begin{algorithmic}[1]
        \Function{FindIndex}{sizes, idx}
            \State candidate \la 0

            \If{candidate < \m\ - 1 \And\ sizes[candidate] <= idx}
                \State candidate++
            \EndIf

            \State \Return candidate
        \EndFunction

        \State

        \Function{RelaxedRadixSearch}{root, key}
            \State node \la\ root
            \State idx \la\ key

            \For{level \la\ root\ts{height} - 1, 1}
                \If{node\ts{sizes}=\nil{}}
                    \State index \la\ (key $\ggg$ (level * x)) \& mask
                    \State node \la\ node[index]
                \Else
                    \State sizes \la\ node\ts{sizes}
                    \State index \la\ \Call{FindIndex}{sizes, idx}
                    \State node \la\ node[index]

                    \If{index != 0}
                        \State idx \la\ idx - sizes[index - 1]
                    \EndIf
                \EndIf
            \EndFor

            \State index \la\ idx \& mask
            \State \Return {node[index]}
        \EndFunction
    \end{algorithmic}

    \caption{Pseudocode of relaxed radix search}
    \label{lst:rrb-tree-relaxed-radix-search}
\end{listing}

\subsection{Concatenation}
The concatenation algorithm used in this project is from \cite{rrb-vector-practical-general-purpose-im-sequence}, which produces a slightly more balanced tree than initially proposed by \cite{efficient-immutable-vectors}. It achieves it by allowing the relaxed tree nodes to have \m{} children instead of $\m{} - 1$.

The algorithm consists of two stages: descending the tree, and then, merging and rebalancing nodes. The time complexity of the presented concatenation algorithm is \bigo{m^2 \cdot{} log_m(n)}.

The recursive function in Listing \ref{lst:rrb-tree-concatenation} descends the tree by selecting the rightmost node of the left tree and the leftmost node of the right tree. If one of the trees is taller than the other, the function descends into a taller tree only until nodes of both trees are at the same level.

When the bottom level with leaf nodes is reached, the function stops descending and starts merging and rebalancing nodes to ensure the \bigo{log_m(n)} bound on the tree height.

\begin{listing}[!ht]
    \begin{algorithmic}[1]
        \Function{Concat}{leftNode, rightNode}
            \If{leftNode\ts{height} > rightNode\ts{height}}
                \State mergedNode \la\ \Call{Concat}{leftNode\ts{last}, rightNode}
                \State \Return \Call{Rebalance}{leftNode\ts{init}, mergedNode, \nil{}}
            \ElsIf{leftNode\ts{height} < rightNode\ts{height}}
                \State mergedNode \la\ \Call{Concat}{leftNode, rightNode\ts{first}}
                \State \Return \Call{Rebalance}{\nil{}, mergedNode, rightNode\ts{tail}}
            \Else
                \State mergedNode \la\ \nil{}

                \If{leftNode\ts{height}=0}
                    \State mergedNode \la\ \Call{Concat}{leftNode, rightNode}
                \Else
                    \If{leftNode\ts{height}=1}
                        \State mergedNode \la\ \Call{Concat}{leftNode\ts{last}, rightNode\ts{first}}
                    \Else
                        \State mergedNode \la\ \Call{Concat}{leftNode\ts{last}, rightNode\ts{first}}
                    \EndIf
                \EndIf

                \State \Return \Call{Rebalance}{leftNode\ts{init}, mergedNode, rightNode\ts{tail}}
            \EndIf
        \EndFunction
    \end{algorithmic}

    \caption{Concatenation algorithm of \treerrb{}}
    \label{lst:rrb-tree-concatenation}
\end{listing}

The \textproc{Rebalance} function accepts three lists of nodes as arguments. The \texttt{left} and \texttt{right} lists constitute all nodes of both trees at the given level except two: the rightmost node of the left tree and the leftmost node of the right tree. Those two nodes are already rebalanced and passed as the \texttt{middle} argument. Three lists are concatenated together into a single \texttt{merged} list.

The goal of rebalancing is to arrange the children of \texttt{merged} nodes in such a way that all nodes except the rightmost branch are filled with values. When the \textproc{Rebalance} function completes re-arranging nodes, it returns a new branch containing rebalanced nodes. Pseudocode for rebalancing can be found in Appendix \ref{lst:rrb-tree-rebalance}.

The rebalancing process is illustrated in Figures \ref{fig:rrb-tree-rebalance-level-0}-\ref{fig:rrb-tree-rebalance}. Note, presented figures exclude parts of trees that are not important for conveying the idea to preserve space.

\begin{figure}[H]
    \colorlet{colorleft}{red!30}
    \colorlet{colorright}{yellow!30}

    \centering
    \footnotesize
    \begin{tikzpicture} [
        node/.style={
            matrix of nodes,
            nodes={draw,minimum width=4mm,minimum height=5mm,anchor=center},
            inner sep=0pt,
            font=\ttfamily,
            text height=1.5ex,
            text depth=.25ex,
            text width=1.5ex,
            text centered,
            nodes in empty cells
        },
        edge/.style={->, shorten >= 4pt},
        edge-dashed/.style={dashed, ->, shorten >= 4pt},
    ]
        \matrix[node, fill=colorleft] (left-1-1) at (current page.north west) { & & & \\ };

        \matrix[node, fill=colorleft] (left-2-4) [below right=8mm and 1mm of left-1-1.south] { & & \\ };
        \node (left-2-3) [left=2mm of left-2-4.west] { \ldots };
        \node (left-2-2) [left=2mm of left-2-3.west] { \ldots };
        \node (left-2-1) [left=2mm of left-2-2.west] { \ldots };

        \draw[edge-dashed, out=225, in=45] (left-1-1-1-1.south) to (left-2-1.north);
        \draw[edge-dashed, out=225, in=45] (left-1-1-1-2.south) to (left-2-2.north);
        \draw[edge-dashed, out=225, in=75] (left-1-1-1-3.south) to (left-2-3.north);
        \draw[edge, out=315, in=90] (left-1-1-1-4.south) to (left-2-4.north);

        \matrix[node, fill=colorleft] (left-3-2) [below left=8mm and 1mm of left-2-4.south] { e & f & g & h \\ };
        \matrix[node, fill=colorleft] (left-3-3) [below right=8mm and 1mm of left-2-4.south] { i & j \\ };
        \matrix[node, fill=colorleft] (left-3-1) [left=2mm of left-3-2.west] { a & b & c & d \\ };

        \draw[edge, out=225, in=45] (left-2-4-1-1.south) to (left-3-1.north);
        \draw[edge, out=225, in=45] (left-2-4-1-2.south) to (left-3-2.north);
        \draw[edge, out=315, in=90] (left-2-4-1-3.south) to (left-3-3.north);

        \matrix[node, fill=colorright] (right-1-1) [right=56mm of left-1-1.east] { & \\ };
        \matrix[node, fill=colorright] (right-2-1) [below left=8mm and 1mm of right-1-1.south] { & & & \\ };
        \node (right-2-2) [right=2mm of right-2-1.east] { \ldots };

        \draw[edge, out=225, in=90] (right-1-1-1-1.south) to (right-2-1.north);
        \draw[edge-dashed, out=315, in=90] (right-1-1-1-2.south) to (right-2-2.north);

        \matrix[node, fill=colorright] (right-3-2) [below left=8mm and 1mm of right-2-1.south] { o & p & q & r \\ };
        \matrix[node, fill=colorright] (right-3-3) [below right=8mm and 1mm of right-2-1.south] { s & t & u & v \\ };
        \matrix[node, fill=colorright] (right-3-1) [left=2mm of right-3-2.west] { k & l & m & n \\ };
        \matrix[node, fill=colorright] (right-3-4) [right=2mm of right-3-3.east] { w & x & y & z \\ };

        \draw[edge, out=225, in=45] (right-2-1-1-1.south) to (right-3-1.north);
        \draw[edge, out=225, in=45] (right-2-1-1-2.south) to (right-3-2.north);
        \draw[edge, out=315, in=135] (right-2-1-1-3.south) to (right-3-3.north);
        \draw[edge, out=315, in=135] (right-2-1-1-4.south) to (right-3-4.north);

        \node[draw, dashed, inner sep=1mm, fit=(left-3-3) (left-3-3) (right-3-1) (left-3-3)] {};
    \end{tikzpicture}

    \caption{Illustration of rebalancing algorithm at level 0}
    \label{fig:rrb-tree-rebalance-level-0}
\end{figure}

Concatenation starts at the bottom of the tree by merging the leaf nodes. The result is a new rebalanced branch with leaves that will be used when rebalancing nodes at level 1.

\begin{figure}[H]
    \colorlet{colorleft}{red!30}
    \colorlet{colorright}{yellow!30}
    \colorlet{colormerged}{orange!30}

    \centering
    \footnotesize
    \begin{tikzpicture} [
        node/.style={
            matrix of nodes,
            nodes={draw,minimum width=4mm,minimum height=5mm,anchor=center},
            inner sep=0pt,
            font=\ttfamily,
            text height=1.5ex,
            text depth=.25ex,
            text width=1.5ex,
            text centered,
            nodes in empty cells
        },
        edge/.style={->, shorten >= 4pt},
        edge-dashed/.style={dashed, ->, shorten >= 4pt},
    ]
        \matrix[node, fill=colorleft] (left-1-1) at (current page.north west) { & & & \\ };

        \matrix[node, fill=colorleft] (left-2-4) [below right=8mm and 1mm of left-1-1.south] { & \\ };
        \node (left-2-3) [left=2mm of left-2-4.west] { \ldots };
        \node (left-2-2) [left=2mm of left-2-3.west] { \ldots };
        \node (left-2-1) [left=2mm of left-2-2.west] { \ldots };

        \draw[edge-dashed, out=225, in=45] (left-1-1-1-1.south) to (left-2-1.north);
        \draw[edge-dashed, out=225, in=45] (left-1-1-1-2.south) to (left-2-2.north);
        \draw[edge-dashed, out=225, in=75] (left-1-1-1-3.south) to (left-2-3.north);
        \draw[edge, out=315, in=90] (left-1-1-1-4.south) to (left-2-4.north);

        \matrix[node, fill=colorleft] (left-3-2) [below right=8mm and 1mm of left-2-4.south] { e & f & g & h \\ };
        \matrix[node, fill=colorleft] (left-3-1) [left=2mm of left-3-2.west] { a & b & c & d \\ };

        \draw[edge, out=225, in=45] (left-2-4-1-1.south) to (left-3-1.north);
        \draw[edge, out=315, in=90] (left-2-4-1-2.south) to (left-3-2.north);

        \matrix[node, fill=colormerged] (merged-1-1) [right=32mm of left-2-4.east] { & \\ };
        \matrix[node, fill=colormerged] (merged-2-1) [below left=8mm and 1mm of merged-1-1.south] { i & j & k & l \\ };
        \matrix[node, fill=colormerged] (merged-2-2) [below right=8mm and 1mm of merged-1-1.south] { m & n \\ };

        \draw[edge, out=225, in=90] (merged-1-1-1-1.south) to (merged-2-1.north);
        \draw[edge, out=315, in=90] (merged-1-1-1-2.south) to (merged-2-2.north);

        \matrix[node, fill=colorright] (right-1-1) [right=72mm of left-1-1.east] { & \\ };
        \matrix[node, fill=colorright] (right-2-1) [below left=8mm and 1mm of right-1-1.south] { & & \\ };
        \node (right-2-2) [right=2mm of right-2-1.east] { \ldots };

        \draw[edge, out=225, in=90] (right-1-1-1-1.south) to (right-2-1.north);
        \draw[edge-dashed, out=315, in=90] (right-1-1-1-2.south) to (right-2-2.north);

        \matrix[node, fill=colorright] (right-3-1) [below left=8mm and 1mm of right-2-1.south] { o & p & q & r \\ };
        \matrix[node, fill=colorright] (right-3-2) [right=2mm of right-3-1.east] { s & t & u & v \\ };
        \matrix[node, fill=colorright] (right-3-3) [right=2mm of right-3-2.east] { w & x & y & z \\ };

        \draw[edge, out=225, in=45] (right-2-1-1-1.south) to (right-3-1.north);
        \draw[edge, out=315, in=135] (right-2-1-1-2.south) to (right-3-2.north);
        \draw[edge, out=315, in=135] (right-2-1-1-3.south) to (right-3-3.north);

        \node[draw, dashed, inner sep=1mm, fit=(merged-1-1) (merged-1-1) (left-2-4) (right-2-1)] {};
    \end{tikzpicture}

    \caption{Illustration of rebalancing algorithm at level 1}
    \label{fig:rrb-tree-rebalance-level-1}
\end{figure}

Rebalancing at level 1 involves the left and right branches together with the newly created branch from the previous step. The algorithm avoids processing nodes where rebalancing is not beneficial. For example, in Figure \ref{fig:rrb-tree-rebalance-level-2} where the \emph{full} nodes from the left-hand side tree are re-used without rebalancing them.

\begin{figure}[H]
    \colorlet{colorleft}{red!30}
    \colorlet{colorright}{yellow!30}
    \colorlet{colormerged}{orange!30}

    \centering
    \footnotesize
    \begin{tikzpicture} [
        node/.style={
            matrix of nodes,
            nodes={draw,minimum width=4mm,minimum height=5mm,anchor=center},
            inner sep=0pt,
            font=\ttfamily,
            text height=1.5ex,
            text depth=.25ex,
            text width=1.5ex,
            text centered,
            nodes in empty cells
        },
        edge/.style={->, shorten >= 4pt},
        edge-dashed/.style={dashed, ->, shorten >= 4pt},
    ]
        \matrix[node, fill=colorleft] (left-1-1) at (current page.north west) { & & \\ };

        \node (left-2-3) [left=2mm of left-2-4.west] { \ldots };
        \node (left-2-2) [left=2mm of left-2-3.west] { \ldots };
        \node (left-2-1) [left=2mm of left-2-2.west] { \ldots };

        \draw[edge-dashed, out=225, in=45] (left-1-1-1-1.south) to (left-2-1.north);
        \draw[edge-dashed, out=225, in=45] (left-1-1-1-2.south) to (left-2-2.north);
        \draw[edge-dashed, out=225, in=75] (left-1-1-1-3.south) to (left-2-3.north);

        \matrix[node, fill=colormerged] (merged-1-1) [right=38mm of left-1-1.east] { & \\ };
        \matrix[node] (merged-2-1) [below left=8mm and 22mm of merged-1-1.south] { & & & \\ };
        \scoped[on background layer] {
            \node[fit=(merged-2-1-1-1), fill=colorleft, inner sep=0pt] {};
            \node[fit=(merged-2-1-1-2), fill=colorleft, inner sep=0pt] {};
            \node[fit=(merged-2-1-1-3), fill=colormerged, inner sep=0pt] {};
            \node[fit=(merged-2-1-1-4), fill=colormerged, inner sep=0pt] {};
        }

        \matrix[node, fill=colormerged] (merged-2-2) [below right=8mm and 22mm of merged-1-1.south] { & & \\ };

        \matrix[node, fill=colorleft] (merged-3-2) [below left=8mm and 1mm of merged-2-1.south] { e & f & g & h \\ };
        \matrix[node, fill=colormerged] (merged-3-3) [below right=8mm and 1mm of merged-2-1.south] { i & j & k & l \\ };
        \matrix[node, fill=colorleft] (merged-3-1) [left=2mm of merged-3-2.west] { a & b & c & d \\ };
        \matrix[node, fill=colormerged] (merged-3-4) [right=2mm of merged-3-3.east] { m & n & o & p \\ };

        \draw[edge, out=225, in=45] (merged-2-1-1-1.south) to (merged-3-1.north);
        \draw[edge, out=225, in=45] (merged-2-1-1-2.south) to (merged-3-2.north);
        \draw[edge, out=315, in=135] (merged-2-1-1-3.south) to (merged-3-3.north);
        \draw[edge, out=315, in=135] (merged-2-1-1-4.south) to (merged-3-4.north);

        \matrix[node, fill=colormerged] (merged-3-6) [below right=8mm and 1mm of merged-2-2.south] { u & v & w & x \\ };
        \matrix[node, fill=colormerged] (merged-3-7) [right=2mm of merged-3-6.east] { y & z \\ };
        \matrix[node, fill=colormerged] (merged-3-5) [left=2mm of merged-3-6.west] { q & r & s & t \\ };

        \draw[edge, out=225, in=90] (merged-1-1-1-1.south) to (merged-2-1.north);
        \draw[edge, out=315, in=90] (merged-1-1-1-2.south) to (merged-2-2.north);

        \draw[edge, out=225, in=45] (merged-2-2-1-1.south) to (merged-3-5.north);
        \draw[edge, out=315, in=135] (merged-2-2-1-2.south) to (merged-3-6.north);
        \draw[edge, out=315, in=135] (merged-2-2-1-3.south) to (merged-3-7.north);

        \matrix[node, fill=colorright] (right-1-1) [right=38mm of merged-1-1.east] { \\ };
        \node (right-2-2) [below right=8mm and 1mm of right-1-1.south] { \ldots };

        \draw[edge-dashed, out=315, in=90] (right-1-1-1-1.south) to (right-2-2.north);

        \node[draw, dashed, inner sep=1mm, fit=(merged-1-1) (merged-1-1) (left-1-1) (right-1-1)] {};
    \end{tikzpicture}

    \caption{Illustration of rebalancing algorithm at level 2}
    \label{fig:rrb-tree-rebalance-level-2}
\end{figure}

Figure \ref{fig:rrb-tree-rebalance} is the last step that rebalances the top-level nodes of the left and right trees, producing a new root. If the root contains only a single child, its child will be promoted to be the root to avoid unnecessary overhead.

\begin{figure}[H]
    \colorlet{colorleft}{red!30}
    \colorlet{colorright}{yellow!30}
    \colorlet{colormerged}{orange!30}

    \centering
    \footnotesize
    \begin{tikzpicture} [
        node/.style={
            matrix of nodes,
            nodes={draw,minimum width=4mm,minimum height=5mm,anchor=center},
            inner sep=0pt,
            font=\ttfamily,
            text height=1.5ex,
            text depth=.25ex,
            text width=1.5ex,
            text centered,
            nodes in empty cells
        },
        edge/.style={->, shorten >= 4pt},
        edge-dashed/.style={dashed, ->, shorten >= 4pt},
    ]
        \matrix[node, fill=colormerged] (root) at (current page.north) { & \\ };

        \matrix[node] (merged-1-1) [below left=8mm and 8mm of root.south] { & & & \\ };
        \scoped[on background layer] {
            \node[fit=(merged-1-1-1-1), fill=colorleft, inner sep=0pt] {};
            \node[fit=(merged-1-1-1-2), fill=colorleft, inner sep=0pt] {};
            \node[fit=(merged-1-1-1-3), fill=colorleft, inner sep=0pt] {};
            \node[fit=(merged-1-1-1-4), fill=colormerged, inner sep=0pt] {};
        }

        \matrix[node] (merged-1-2) [below right=8mm and 8mm of root.south] { & \\ };
        \scoped[on background layer] {
            \node[fit=(merged-1-2-1-1), fill=colormerged, inner sep=0pt] {};
            \node[fit=(merged-1-2-1-2), fill=colorright, inner sep=0pt] {};
        }

        \draw[edge, out=270, in=75] (root-1-1.south) to (merged-1-1.north);
        \draw[edge, out=270, in=115] (root-1-2.south) to (merged-1-2.north);

        \matrix[node] (merged-2-1) [below left=8mm and 1mm of merged-1-1.south] { & & & \\ };
        \scoped[on background layer] {
            \node[fit=(merged-2-1-1-1), fill=colorleft, inner sep=0pt] {};
            \node[fit=(merged-2-1-1-2), fill=colorleft, inner sep=0pt] {};
            \node[fit=(merged-2-1-1-3), fill=colormerged, inner sep=0pt] {};
            \node[fit=(merged-2-1-1-4), fill=colormerged, inner sep=0pt] {};
        }

        \matrix[node, fill=colormerged] (merged-2-2) [below=8mm of merged-1-2.south] { & & \\ };

        \node (merged-left-3) [left=2mm of merged-2-1.west] { \ldots };
        \node (merged-left-2) [left=2mm of merged-left-3.west] { \ldots };
        \node (merged-left-1) [left=2mm of merged-left-2.west] { \ldots };

        \draw[edge-dashed, out=205, in=75] (merged-1-1-1-1.south) to (merged-left-1.north);
        \draw[edge-dashed, out=205, in=55] (merged-1-1-1-2.south) to (merged-left-2.north);
        \draw[edge-dashed, out=205, in=45] (merged-1-1-1-3.south) to (merged-left-3.north);

        \node (merged-right-1) [right=2mm of merged-2-2.east] { \ldots };
        \draw[edge-dashed, out=315, in=105] (merged-1-2-1-2.south) to (merged-right-1.north);

        \matrix[node, fill=colorleft] (merged-3-2) [below left=8mm and 10mm of merged-2-1.south] { e & f & g & h \\ };
        \matrix[node, fill=colormerged] (merged-3-3) [below=8mm of merged-2-1.south] { i & j & k & l \\ };
        \matrix[node, fill=colorleft] (merged-3-1) [left=2mm of merged-3-2.west] { a & b & c & d \\ };
        \matrix[node, fill=colormerged] (merged-3-4) [right=2mm of merged-3-3.east] { m & n & o & p \\ };

        \draw[edge, out=205, in=45] (merged-2-1-1-1.south) to (merged-3-1.north);
        \draw[edge, out=205, in=45] (merged-2-1-1-2.south) to (merged-3-2.north);
        \draw[edge, out=270, in=90] (merged-2-1-1-3.south) to (merged-3-3.north);
        \draw[edge, out=270, in=90] (merged-2-1-1-4.south) to (merged-3-4.north);

        \matrix[node, fill=colormerged] (merged-3-5) [below=8mm of merged-2-2.south] { q & r & s & t \\ };
        \matrix[node, fill=colormerged] (merged-3-6) [right=2mm of merged-3-5.east] { u & v & w & x \\ };
        \matrix[node, fill=colormerged] (merged-3-7) [right=2mm of merged-3-6.east] { y & z \\ };

        \draw[edge, out=225, in=90] (merged-1-1-1-4.south) to (merged-2-1.north);
        \draw[edge, out=315, in=90] (merged-1-2-1-1.south) to (merged-2-2.north);

        \draw[edge, out=270, in=90] (merged-2-2-1-1.south) to (merged-3-5.north);
        \draw[edge, out=270, in=135] (merged-2-2-1-2.south) to (merged-3-6.north);
        \draw[edge, out=315, in=135] (merged-2-2-1-3.south) to (merged-3-7.north);
    \end{tikzpicture}

    \caption{Illustration of rebalancing algorithm: the rebalanced tree}
    \label{fig:rrb-tree-rebalance}
\end{figure}

The formal description and analysis of concatenation algorithm and its implementation is thoroughly presented in \cite{improving-performance-through-transience}.

\section{Transience}
% TODO

\chapter{Definition of RRB-Tree in Rust}

In order to define a persistent vector in Rust, we first need to understand how different parts of it are connected to each other in the computer memory, and which Rust constructs are the most suitable for their representation. 

The backbone of a confluently persistent vector is \rrbtree, with auxiliary properties such as a tail, size, and height. \rrbtree\ consists of infinitely nested nodes, which form a directed acyclic graph in memory \todo{add figure}. 

Since the data structure can be arbitrarily large, the Rust compiler is not able to reliably measure its size during compilation. Hence, the static memory allocation on the stack is not possible without additional constraints, such as fixed capacity. 

In fact, the Rust compiler will abort the compilation if a \emph{recursive type} is encountered. A recursive type is a type, where a value can have as part of itself another value of the same type, such as a \rrbtree\ node. The solution is to use dynamic memory allocation instead of static. Rust provides a special type of pointers for this purpose, such as Box, Arc, etc. 

As \rrbtree\ employs structural sharing, several tree instances might point to the same subtrees. In other words, one node might be referenced by several parent nodes simultaneously. Even though Box enables recursive types, it does not allow shared ownership of the underlying value. A smart pointer which supports a notion of shared ownership is known as \emph{Rc} or reference counted pointer. 

Even though Box enables recursive types, it does not allow shared ownership of the underlying value. A smart pointer which supports a notion of shared ownership is known as \emph{Rc<T>} or reference counted pointer. 

\todo{introduction: figure out a notation for talking about Rust types}
\todo{introduction: define what is ownership in Rust}

The standard library also offers an \emph{Arc<T>} type, which is very similar to Rc<T> with the difference that it uses atomic operations to synchronize accesses to its reference counts. This can make Arc<T> a little more expensive during run time, but it enables threads to share a value safely. 

\todo{project: PVec does not seem to implement Send + Sync methods}
\todo{performance: look into other perf measurement tools, such as alloc-counter}

As a persistent vector is intended to be a part of a library, Arc would be a better choice as it creates a foundation for the data structure to be used in the multithreaded environment. 

\todo{Maybe insert a diagram here of how arc looks \& works internally}

In the nutshell, a reference counting pointer allows shared ownership of the object wrapped into it. Every time a potential owner needs a reference to the value, the pointer is cloned instead of the underlying object. On each clone the reference count is incremented, and decremented every time a pointer goes out of the scope. If reference count reaches 0, the underlying value is destroyed. 

Rust’s Arc conforms to the ownership and borrowing rules as well, but they are enforced during run time rather than compile time. It allows shared immutable access to the value as well as regular reference does, and permits unique mutable access only if other reference do not exist. As reference count is expected to be more than 1, it features methods which copy the underlying value if there are other references. 

\todo{Example of Arc::make\_mut}

\todo{Transience as a language feature}

\chapter{Performance evaluation}

In this chapter, I will introduce a methodology for performance evaluation of \rrbvec{}, \pvec{} and their variants, in comparison to implementations from \imrsvec{} and the Rust's standard library. We will look at the details of these three stages:

\begin{itemize}
    \item First, a methodology for collecting reliable measurements. 
    \item Then identifying directions for performance comparisons. 
    \item Finally, defining benchmarks. 
\end{itemize}

To begin with, I will present a notion of benchmarking framework, before diving into details of specific profiling tests. 

\section{Methodology}
Even though \bigo{} \todo{Use a word representation of Big-O} is a useful tool for reasoning about performance in theory, it often is not accurate enough for evaluating real-world performance. First, it disregards constant factors as they are not significant for the growth rate of functions. Second, it does not consider the architecture of CPU and memory \todo{ref}, which indeed influences performance. Furthermore, it also applies to software, such as operating systems, schedulers, virtual machines, et cetera. Hence, often algorithms which are expected to be equally fast based on \bigo{}, may differ substantially in real-world performance. 

This leads us to a need for an experimental performance analysis approach, which would involve executing tests on the actual hardware and software. This, however, introduces another set of unique challenges. For instance, depending on the workload, operating systems may allocate more resources for high demanding tasks, by reducing runtime for others \todo{ref}. Such non-deterministic behavior may lead to profiling results which vary from run to run significantly, which defeats the purpose of having them. 

Benchmarking frameworks were introduced to solve those problems. They are designed to get stable measurements by executing the same test thousands of times. Some of them, such as criterion \todo{ref} for Haskell and Rust, JMH \todo{ref} for Java, and ScalaMeter for Scala, introduce statistical methods for the detection and elimination of exceptionally different runs, known as outliers. 

\subsection{Benchmarking frameworks for Rust}

There are several benchmarking frameworks available for Rust, and unfortunately, none of them have reached a stable release yet. However, some of them are being actively used in the Rust community and proven to produce reliable results.  

In our case, there are several criteria which a good framework has to meet:

\begin{itemize}
    \item Collects multiple samples where each sample consists of multiple runs to ensure consistent results. 
    \item Detection and elimination of outliers.     
    \item A way for setting up a benchmark before each run. 
    \item A way for preventing compiler optimizing benchmark code away.     
\end{itemize}

\subsubsection*{Rust's benchmark tests}
The Rust's testing framework provides an experimental feature which enables developers to write test benchmarks. Those benchmarks are executed thousands of times until results are stabilized. Also, it provides a black-box function \footnote{Black box function contains inline assembly instructions, which compiler cannot make any assumptions about. Hence, it prevents the compiler from optimizing the code which otherwise would be considered "dead" or unused.} which is opaque for the compiler. 

However, it does not detect and eliminate anomalies. It also does not provide APIs for setup routines, which makes it impossible to create benchmarks which rely on certain preconditions.

% ToDo: consider talking about measurement of time for dropping items
\subsubsection*{Criterion for Rust}
Criterion for Rust \todo{ref}, not to be confused with Criterion for Haskell \todo{ref}, is a powerful and statistically rigorous tool for profiling code. It features outlier elimination, setup routines, and is capable to generate graphs provided that gnuplot is installed \todo{ref}. it is available for the stable Rust compiler. Thus, Criterion was chosen as a benchmarking framework for this project. 

\subsection{Execution environment}
All benchmarks were executed on a computer \todo{table with hardware, software} with a quad-core Intel Core i5-6600 processor with hyperthreading, 16GB of DDR4 RAM and 250GB solid-state drive. The operating system of the choice is Ubuntu 18.04 with nightly Rust compiler version \todo{Rust version}.

\subsection{Configuration and input size}
\todo{section}

\section{Benchmarking directions}
% ToDo: you have completely forgotten about branching factor, and it being a standalone feature
% ToDo: you haven't talked about input size and configuration of benchmarks (what will be the input size). You can steal input numbers from scala paper, and then tweak them to run for reasonable amount of time. 

To understand how effective certain optimizations are, we need to evaluate various configurations of the persistent vector. Additionally, implementations from \imrsvec{} and the standard library will be tested too. All vector variants are specified in table \todo{ref}. 

% ToDo: try to name types after the name of types in actual code, otherwise thins will get very confusing very fast. Also, I don't think it is important to show distinction between Rc / Arc flavors of implementations, because it just introduces more confusion. 
\begin{center}
\begin{tabular} { |l| p{10cm} | }
    \hline
    STD Vec & Vec from the Rust's standard library. \\ \hline
    \rbvec{} & \rbtree{} based vector. \\ \hline
    \rrbvec{} & \rrbtree{} based vector. \\ \hline
    \pvec{} & \rrbtree{} based vector with dynamic internal representation. \\ \hline
    \imrsvec{} & \rrbtree{} based vector from third party library \imrsvec{}. \\
    \hline        
\end{tabular}
\end{center}

% \begin{wip}
    % For the purpose of comparison, the measurements are taken against the standard vector, as well as the \rrbtree based persistent vector implementation from the third party library named \emph{im-rs}, which has been introduced at the time of writing this paper. The persistent vector presented in this work, has been evaluated using both non-atomic and atomic reference counted pointers, as well as with and without small sized vector based optimization. 
% \end{wip}

In general, benchmarks described below could be categorized into two groups:

\begin{itemize}
    \item First, serial benchmarks for profiling core operations in a sequential environment. They will be executed both against threadsafe and non-threadsafe variants of the vector. 
    \item Second, concurrent benchmarks which will be executed against only thread-safe variants of the vector. The goal is to check whether there are benefits of using fast split and combine operations of \rrbvec{}.
\end{itemize}


\subsection{RB and RRB Vectors}
As relaxed balanced tree is not perfectly balanced and involves the use of size tables for the radix search, it is expected to be somewhat slower in all core operations. This, however, is not true from the perspective of asymptotic analysis, where constant factors are neglected. The goal of benchmarks, in this case, is to reveal the overhead induced by relaxed nodes. 

Before each benchmark run, an instance of \rrbvec{} will be prepared by concatenating pseudo-random small vectors together. The amount of relaxed nodes is in part affected by the size of the vector. Both threadsafe and non-threadsafe variants will be compared. 

% ToDo: how to consistenly generate RRB Vectors using pseudo randomly sized small vectors? Can you get access to benchmarks from paper anywhere? 
% ToDo: add list of benchmarks which will be evaluating different core operations. 

\subsection{Unique access or transience}
While \rrbvec{} performs very well as a persistent data structure, it is not very optimal when properties of persistence are not required. An example is a function which creates and returns an instance of \rrbvec{}, where all versions except the returned one are disregarded.

Luckily, the persistent vector presented in this project takes advantage of Rust's compiler capabilities of tracking object aliasing. Thus, it avoids redundant copying on mutation if the given object is uniquely accessed. This behavior is somewhat similar to transience in the Clojure's persistent vector, but not entirely identical \todo{see reference}. 

In Rust, non-transient, persistent behavior can be enforced by cloning the object before performing a mutation. The goal is to measure the overhead of using clone operation in the persistent vector. 

\subsection{Dynamic internal representation}
As one of the suggested optimizations in \todo{ref to scala paper}, a standard vector can be used to improve the performance of small-sized \rrbvec{}. The size recommended for using the standard vector representation is 4096. However, dynamically switching representation during runtime comes at a cost, which potentially may offset the benefits. 

The purpose of profiling this optimization is to understand whether it improves performance in practice, and in which use cases. The range of problem sizes will include small values. 

\todo{describe an experiment; i.e. how many threads, split factor, problem sizes, etc.}
\todo{how number of threads is configured in rayon: 1, 2, 4, 8, 16, 32 and 64}
\todo{which experiments will you run?}

\subsection{Memory overhead}
\todo{tbd}

\subsection{Rc vs Arc}
Since atomic reference-counted pointers are claimed to introduce additional overhead in comparison to their non-threadsafe counterpart, the goal is to check how significant is the difference. 

As a part of all sequential benchmarks, both threadsafe and non-threadsafe variants will be evaluated. Rc based flavor will not be present in parallel benchmarks, as the Rust's compiler prevents non-threadsafe types being used across threads. 

\section{Parallel vector}
One of the claims is that \rrbvec{} is very efficient when it comes to split and concatenate operations. The data parallelism frameworks, such as Rayon \todo{ref}, Cilk \todo{ref}, and Scala's parallel collections \todo{ref}, split the work into smaller chunks to ensure good parallelism. Thus, fast split and concat operations are critical for optimal performance. 

In this section, we will first take a look at how Rayon splits and distributes the work across threads, as well as available configuration parameters. Then, section \todo{} introduces tests for benchmarking the overall performance of persistent and standard vectors:

\begin{itemize}    
    \item Processing a vector of integers.    
    \item Check if a word is a palindrome.         
\end{itemize}

All the tests will be executed on 1, 2, 4, 8, and 16 threads. 

Unlike the measurements presented from the sequential benchmarks, the parallel ones inlude the run time of both vector operations and Rayon. As the objective is the overall performance comparison, this is considered to be an acceptable tradeoff. 

The results will be used to evaluate the effectivness of following optimizations:
\begin{itemize}
    \item The effect of relaxed concat and split operations of \rrbvec{} on the overall performance. 
    \item Dynamic internal representation of \pvec{}.     
\end{itemize}

\subsection{Rayon}
The idea of Rayon, a data parallelism library for Rust, is to turn sequential code into parallel with as little work as possible. Loops and iterators are often used to process collections sequentially. Rayon on the other hand, offers a potentially more efficient alternative to them in the form of parallel iterators. It takes advantage of modern processors, by dividing the work between available cores when it is considered to be beneficial. 

\paragraph*{Parallel iterators}

\begin{figure}[!htbp] 
    \centering

    \begin{minted}{rust}
        // sequential iterator
        vec![1, 2, 3]
            .into_iter()
            .for_each(|x| println!("{}", x));

        // rayon's parallel iterator
        vec![1, 2, 3]
            .into_par_iter()
            .for_each(|x| println!("{}", x));
    \end{minted}

    \caption{Example of using sequential and parallel iterators.}
    \label{fig:par-iter-example}
\end{figure}

The power of iterators in Rust is in the operations which it provides over its elements. Structures which implement those operations are called combinators. They can be chained and the result of execution is passed from one combinator to another. 

Parallel iterators expose similar set of operators, even though not entirely identical. As iterators process values sequentially, there is a set of combinators which expect values to be emitted in particular order. As the parallel iterators are designed to process data in any order, inherentely sequential combinators are simply not applicable. Thus, Rayon might be not suitable for algorithms relying on the sequential order of execution.

Another limitation which parallel iterators impose, is that type of values which it works with have to implement the \emph{Send} trait. It means using non-threadsafe types such as \emph{Rc} in combination with Rayon is prohibited. 

\paragraph*{Work splitting}
\begin{figure}[!htbp]
    \centering

    \begin{tikzpicture}[
        font=\ttfamily,
        array/.style={
            matrix of nodes,
            nodes={draw, minimum size=7mm, fill=green!30},
            column sep=-\pgflinewidth, 
            row sep=0.5mm, 
            nodes in empty cells,        
            row 1 column 1/.style={nodes={draw}}
        }]
                
        \matrix[array] (array) { 
            1 & 2 & 3 & 4 & 5 & 6 & 7 & 8 & 9 & 10 & 11 & 12 \\
        };            
        
        \draw[|-|]([yshift=-4mm,xshift=1mm]array-1-1.south west) -- node[above,font=\tiny,outer sep=0mm] {12} ([yshift=-4mm,xshift=-1mm]array-1-12.south east);

        \draw[|-|]([yshift=-8mm,xshift=1mm]array-1-1.south west) -- node[above,font=\tiny,outer sep=0mm] {6} ([yshift=-8mm,xshift=-1mm]array-1-6.south east);
        \draw[|-|]([yshift=-8mm,xshift=1mm]array-1-7.south west) -- node[above,font=\tiny,outer sep=0mm] {6} ([yshift=-8mm,xshift=-1mm]array-1-12.south east);

        \draw[|-|]([yshift=-12mm,xshift=1mm]array-1-1.south west) -- node[above,font=\tiny,outer sep=0mm] {3} ([yshift=-12mm,xshift=-1mm]array-1-3.south east);
        \draw[|-|]([yshift=-12mm,xshift=1mm]array-1-4.south west) -- node[above,font=\tiny,outer sep=0mm] {3} ([yshift=-12mm,xshift=-1mm]array-1-6.south east);
        \draw[loosely dotted]([yshift=-12mm,xshift=1mm]array-1-7.south west) -- ([yshift=-12mm,xshift=-1mm]array-1-12.south east);

        \draw[|-|]([yshift=-16mm,xshift=1mm]array-1-1.south west) -- node[above,font=\tiny,outer sep=0mm] {2} ([yshift=-16mm,xshift=-1mm]array-1-2.south east);
        \draw[|-|]([yshift=-16mm,xshift=1mm]array-1-3.south west) -- node[above,font=\tiny,outer sep=0mm] {1} ([yshift=-16mm,xshift=-1mm]array-1-3.south east);
        \draw[loosely dotted]([yshift=-16mm,xshift=1mm]array-1-4.south west) -- ([yshift=-16mm,xshift=-1mm]array-1-12.south east);

        \draw[|-|]([yshift=-20mm,xshift=1mm]array-1-1.south west) -- node[above,font=\tiny,outer sep=0mm] {1} ([yshift=-20mm,xshift=-1mm]array-1-1.south east);
        \draw[|-|]([yshift=-20mm,xshift=1mm]array-1-2.south west) -- node[above,font=\tiny,outer sep=0mm] {1} ([yshift=-20mm,xshift=-1mm]array-1-2.south east);
        \draw[loosely dotted]([yshift=-20mm,xshift=1mm]array-1-2.south west) -- ([yshift=-20mm,xshift=-1mm]array-1-12.south east);    
        
        \draw ([xshift=1mm]array-1-12.east)--++(0:3mm) node [right]{ Vector };
    \end{tikzpicture}

    \caption{Visualization of work splitting in Rayon.}
    \label{fig:rayon-work-splitting}
\end{figure}

One of the Rayon's components, fork/join framework, is responsible for dividing and distributing the work between threads. When parallel iterator receives values from a collection like vector, Rayon attempts to repeatedely divide the work into chunks among threads until the chunk is small enough for a single thread. See an example in figure \ref{fig:rayon-work-splitting}.

As demonstrated in figure \ref{fig:rayon-join}, the work is \emph{potentially} divided between two threads by calling \emph{rayon::join}, which accepts two clojures. Rayon decides whether it is beneficial to parallelize the work, depending on the count of available threads, the split factor and the work load. If the problem is small enough, it is solved sequentially. Otherwise, it is subdivided into smaller parts. When both clojures finish working, the results are combined and returned to the caller. 

The size of the work chunk, or the \emph{split factor}, can be controlled by two operations available for \emph{IndexedParallelIterator}, namely \emph{with\_min\_len} and \emph{with\_max\_len}. The use of combinators which may affect the size of a collection, such as \emph{filter}, returns a \emph{ParallelIterator} which does not support configuration of the \emph{split factor}. 

\begin{figure}[!htbp]
    \centering

    \begin{minted}{rust}
        rayon::join(
            || do_something(...),
            || do_something_else(...)
        );
    \end{minted}
    
    \caption{An example of using rayon's join.}
    \label{fig:rayon-join}
\end{figure}

By default, the count of threads allocated by Rayon is equal to the number of cores available in the system. To observe how the thread count affects the performance, Rayon's threadpool will be configured to work with 2, 4, 8, and 16 threads. 

\paragraph*{Load balancing}
In perfect world the chunks of work split between threads take the same amount of time to process. In reality this is often not the case, resulting in some threads idling. In Rayon, each thread has a queue of work attached to it. It keeps processing the queue until it becomes empty. In order to avoid idling, the thread which has finished processing its queue can steal work from another thread. This technique is know as work stealing and is used as the main mechnasim for work distribution in Rayon. 

\paragraph*{Computation stages}
Computation stages of both "Processing a vector of integers" and "Check if a word is a palindrome" benchmarks, could be described in three steps. 
\begin{itemize}
    \item First, split the work between threads. 
    \item Then process the chunk of work sequentially. 
    \item Finally, combine and return the results:    
\end{itemize}

The final step can be subdivided further:
\begin{itemize}
    \item Collect individual items into a vector using the parallel \emph{Fold} combinator. 
    \item Reduce emitted vectors into a single one using the \emph{Reduce} combinator. 
\end{itemize}

% Explain a little bit further what fold and reduce operators do, and what is the difference between them

\begin{figure}[!htbp]
    \centering

    \begin{minted}{rust}
        let result = parallel_iterator
            .fold(Vec::new, |mut vec, x| {
                vec.push(x);
                vec
            })
            .reduce(Vec::new, |mut vec1, mut vec2| {
                vec1.append(&mut vec2);
                vec1
            });
    \end{minted}
    
    \caption{Collecting items of parallel iterator.}
    \label{fig:fold-reduce}
\end{figure}

% You can depict this part as a tree (you already have an example)

%   Thread 1       Thread 2       Thread 3       Thread 4 
% |1| |2| |3|     |4| |5| |6|    |7| |8| |9|   |10| |11| |12|

% increment all values 
% |2| |3| |4|     |5| |6| |7|    |8| |9| |10|  |11| |12| |13|

% fold
%  |2, 3, 4|       |5, 6, 7|      |8, 9, 10|    |11, 12, 13|

% reduce
% |1, 2, 3, 4, 5, 6, 7, 8, 9, 10, 11, 12|


\subsection{Benchmarks}
The benchmarks were executed against following vector implementations: Vec, RbVec, RrbVec, and PVec, where RbVec, RrbVec, and PVec are based on the threadsafe reference-counted pointer -- \emph{Arc}. The \imrsvec{} is not included because it does not implement the Rayon's \emph{IntoParallelIterator} trait, which makes its evaluation irrelevant. 

Benchmarks have been parameterized over two dimensions: the vector size and the number of threads. To see whether parallelism is beneficial, each benchmark has an analogous, sequential implementation executed on a single thread. 

\subsubsection*{Processing a vector of integers}
This benchmark is subdivided into two separate tests: 
\begin{itemize}
    \item Incrementing each item of a vector.
    \item Filtering out odd integers. 
\end{itemize}

The setup routine of this benchmark is identical for both tests, which generates a vector of [0, N] integers, where \emph{N} is the problem size. As soon as processing is complete, parallel iterator is reduced to a vector. 

\paragraph*{Check if a word is a palindrome}
For each word in the given file, check if it is a palindrome. Words are provided in a text file separated by spaces. The result of execution is a list of pairs, where the first item is the word and the second is a flag indicating whether it is a palindrome or not. 

% how the file is loaded and what is the setup routine
The contents of the file are loaded into memory once, and then before each run they are copied over to a new vector within the setup routine. A vector is later passed to the test routine, which in a turn, converts it to a parallel iterator. 

% what is the range of problem sizes 
The total count of words in the file is 370103. Even though all words are available, the setup routine only returns N words, where N is the problem size. The range of problems for this benchmark is [10000, 370103]. 

\section{Presentation of results}
The results show the mean measured time for each function as the input increases. 

\section{Reproducing results}
To execute sequential benchmarks do this. In order to execute sequential benchmarks using \emph{Arc} use this. If you want to run parallel benchmarks, do this and this. Criterion can as well generate reports with charts using gnuplot, but you need to make sure that it is installed on the system. You need to be located in the root directory of the project. 

Executing sequential benchmarks:
\begin{figure}[!htbp]
    \centering

    \begin{minted}{bash}
        # benches of the non-threadsafe implementation
        cargo bench

        # benches of the threadsafe implementation
        cargo bench --features=arc
    \end{minted}
    
    \caption{Example 1.}
    \label{fig:sequential-benches}
\end{figure}

Executing parallel benchmarks:
\begin{figure}[!htbp]
    \centering

    \begin{minted}{bash}
        cargo bench --features=arc,rayon-iter
    \end{minted}
    
    \caption{Example 2.}
    \label{fig:parallel-benches}
\end{figure}

You can find the results in the next directory: project/target/criterion/. If you had gnuplot installed, then the report with generated charts can be found here: project/target/criterion/report/index.html. 
\newcommand{\balanced}{}
\newcommand{\standard}{\emph{(s)}}
\newcommand{\relaxed}{\emph{(r)}}

\chapter{Benchmarks and results}
In this chapter, we will take a look at the results of the sequential and parallel benchmarks, and discuss whether proposed\todo{ref} performance optimizations are effective.

The sequential benchmark results are subdivided per the core operation and listed under the \ref{sec:perf-seq} section. Performance of the threadsafe implementation is evaluated separately.

The parallel benchmark results in section \ref{sec:perf-par} are not discussed by operation, as they focus on the overall performance comparison of vectors rather than particular operation in isolation.

The results are discussed to address the following points:
\begin{itemize}
    \item Effect of \rrbtree{} relaxation on concatenation, splitting, and other core operations of \rrbvec{} and \pvec{}.
    \item The impact of the unique access optimization on the performance of all vector implementations.
    \item Effectiveness of the dynamic representation.
\end{itemize}

\paragraph{Reading notes}
Implementations that are prefixed with \relaxed{} in the figure legend were configured to use the relaxed \emph{rb} tree in the benchmark. \standard{} stands for standard and will be applied only to \pvec{} when it is flat. If not specified, the vector is backed by the balanced \rbtree{}.

\section{Sequential benchmark results}
% > =================================================================================
Each benchmark described in this section focuses on a particular core operation of a vector. To avoid ambiguous results, each test exercises only one operation at a time. Operations that modify vector, such as push, will have a complementary version of the benchmark, which also uses the clone operation. This is necessary for comparison of the path copying and naive algorithms used in the tree-based and standard vectors correspondingly.

\todo{Put this table where you first define core operations}
The following operations were evaluated for vector implementations in \ref{tab:vec-implementations}:
\begin{table}[!htbp]
    \centering

    \begin{tabular} { |l| p{10cm} | }
        \hline
        Indexing & Accessing vector values. \\ \hline
        Updating & Updating existing values. \\ \hline
        Pushing & Adding new values to the end of a vector. \\ \hline
        Popping & Removing values at the end of a vector. \\ \hline
        Appending & Concatenating values of one vector to another. \\ \hline
        Splitting & Slicing one vector into two at a given position. \\ \hline
    \end{tabular}

    \label{tab:vec-core-operations}
    \caption{A table of core operations.}
\end{table}

\paragraph{Benchmark structure}
Some benchmarks depend on preconditions. For example, to test indexing, we first need to create a vector with values. Since building a vector instance is not a part of that test, it happens in the setup routine. Hence, benchmarks with preconditions are executed in two steps: setup and the actual test.

\paragraph{Benchmarking dimensions}
Every benchmark for a core operation is parameterized over the vector size. By providing different arguments, we can observe how the performance of vectors is affected in response. This is especially insightful for the tree-based implementations, where the size of the vector influences the height of the tree, which has a negative impact on performance. The output of a benchmark for a given size is the mean runtime in \millis{}.
% < =================================================================================

\label{sec:perf-seq}
This section contains performance numbers for the non-threadsafe implementations of \stdvec{}, \rbvec{}, \rrbvec{}, \pvec{}, and \imrsvec{}. Threadsafe variants are evaluated and discussed separately in section \ref{sec:perf-rc-vs-arc}.

\subsection{Indexing}
% > =================================================================================
In this section, we will define benchmarks for accessing values in three different configurations, which model the most common ways of working with vector:

\begin{itemize}
    \item Sequentially accessing values by index and iterator.
    \item Accessing values at randomly generated indices.
\end{itemize}

Use cases listed above address two objectives: first, how much overhead relaxed nodes of \rrbtree{} introduce in comparison to \rbtree{}, and second, the efficiency of the dynamic representation in \pvec{}.

The indexing benchmarks share the same setup routine for generating a vector. Balanced tree-based vectors are created by pushing 64-bit integers, while the relaxed types are generated by concatenating vectors together.

The vector size is passed as an argument and falls into the range of \range{[20, \mega{1}]}.
% < =================================================================================

Figures for the index operation are separated by sequential and random access, where the sequential benchmark results are subdivided into the index and iterator figures.

\subsubsection*{Index sequentially}
% > =================================================================================
The benchmark with access by index loops over the array of \range{[0, N)} indices, and reads values from a vector at corresponding positions. Values are read by using immutable references without taking ownership of them.

The iterators test reads the contents of the tree-based vectors by chunks, rather than by individual values. Additionally, iterator takes ownership of values instead of borrowing them. Due to these differences, the results of this benchmark will not be compared to the access by index.
% < =================================================================================

\begin{figure}[!htbp]

    \center
    \begin{adjustbox}{width=\textwidth}
    \begin{tikzpicture}
        \tikzstyle{every node}=[
            font=\scriptsize,
            inner sep=2pt,
            outer sep=0pt
        ]

        \pgfplotstableread[col sep=comma]{data/index_sequentially/im-rs-vector-balanced.csv}\idxseqimrsvectorbalanced;
        \pgfplotstableread[col sep=comma]{data/index_sequentially/im-rs-vector-relaxed.csv}\idxseqimrsvectorrelaxed;
        \pgfplotstableread[col sep=comma]{data/index_sequentially/pvec-rrbvec-balanced.csv}\idxseqpvecbalanced;
        \pgfplotstableread[col sep=comma]{data/index_sequentially/pvec-rrbvec-relaxed.csv}\idxseqpvecrelaxed;
        \pgfplotstableread[col sep=comma]{data/index_sequentially/pvec-std.csv}\idxseqpvecstd;
        \pgfplotstableread[col sep=comma]{data/index_sequentially/rbvec.csv}\idxseqrbvecbalanced;
        \pgfplotstableread[col sep=comma]{data/index_sequentially/rrbvec.csv}\idxseqrrbvecrelaxed;
        \pgfplotstableread[col sep=comma]{data/index_sequentially/std-vec.csv}\idxseqstdvector;

        \pgfplotstableread[col sep=comma]{data/iterator_next/im-rs-vector-balanced.csv}\itrnextimrsvectorbalanced;
        \pgfplotstableread[col sep=comma]{data/iterator_next/im-rs-vector-relaxed.csv}\itrnextimrsvectorrelaxed;
        \pgfplotstableread[col sep=comma]{data/iterator_next/pvec-rrbvec-balanced.csv}\itrnextpvecbalanced;
        \pgfplotstableread[col sep=comma]{data/iterator_next/pvec-rrbvec-relaxed.csv}\itrnextpvecrelaxed;
        \pgfplotstableread[col sep=comma]{data/iterator_next/pvec-std.csv}\itrnextpvecstd;
        \pgfplotstableread[col sep=comma]{data/iterator_next/rbvec.csv}\itrnextrbvecbalanced;
        \pgfplotstableread[col sep=comma]{data/iterator_next/rrbvec.csv}\itrnextrrbvecrelaxed;
        \pgfplotstableread[col sep=comma]{data/iterator_next/std-vec.csv}\itrnextstdvector;

        \begin{groupplot}[
            group style={group size=2 by 1, horizontal sep=56pt,},
            xlabel={Vector size (log scale)},
            ylabel={Mean time (log scale) [\millis{}]},
            yticklabels={0, \micros{0.01}, \micros{0.1}, \micros{1}, 0.01, 0.1, 1, 10, 100},
            xticklabels={0, 10, 100, \kilo{1}, \kilo{10}, \kilo{100}, \mega{1}},
            ymajorgrids=true,
            xmajorgrids=true,
            grid style=dashed,
        ]
            \nextgroupplot[
                xmode=log,
                ymode=log,
                title={Index sequentially},
                legend columns=4,
                legend style={
                    at={(1.12,-0.2)},
                    anchor=north
                }
            ]
            \addplot[ultra thin, color=morange, mark=*, mark size=1.2pt,] table {\idxseqstdvector};
            \addplot[ultra thin, color=mred, mark=*, mark size=1.2pt,] table {\idxseqrbvecbalanced};
            \addplot[ultra thin, color=mred, mark=pentagon, mark size=1.6pt,] table {\idxseqrrbvecrelaxed};
            \addplot[ultra thin, color=mgreen, mark=*, mark size=1.2pt,] table {\idxseqpvecstd};
            \addplot[ultra thin, color=mgreen, mark=square, mark size=1.6pt,] table {\idxseqpvecbalanced};
            \addplot[ultra thin, color=mgreen, mark=diamond*, mark size=1.2pt,] table {\idxseqpvecrelaxed};
            \addplot[ultra thin, color=mpurple, mark=pentagon*, mark size=1.2pt,] table {\idxseqimrsvectorbalanced};
            \addplot[ultra thin, color=mpurple, mark=square, mark size=1.6pt,] table {\idxseqimrsvectorrelaxed};
            \legend{\stdvec{}, \rbvec{}, \rrbvec{} \relaxed{}, \pvec{} \standard{}, \pvec{} \balanced{}, \pvec{} \relaxed{}, \imrsvec{}, \imrsvec{} \relaxed{}}

            \nextgroupplot[xmode=log,ymode=log,title={Iterator}]
            \addplot[ultra thin, color=morange, mark=*, mark size=1.2pt,] table {\itrnextstdvector};
            \addplot[ultra thin, color=mred, mark=*, mark size=1.2pt,] table {\itrnextrbvecbalanced};
            \addplot[ultra thin, color=mred, mark=pentagon, mark size=1.6pt,] table {\itrnextrrbvecrelaxed};
            \addplot[ultra thin, color=mgreen, mark=*, mark size=1.2pt,] table {\itrnextpvecstd};
            \addplot[ultra thin, color=mgreen, mark=square, mark size=1.6pt,] table {\itrnextpvecbalanced};
            \addplot[ultra thin, color=mgreen, mark=diamond*, mark size=1.2pt,] table {\itrnextpvecrelaxed};
            \addplot[ultra thin, color=mpurple, mark=pentagon*, mark size=1.2pt,] table {\itrnextimrsvectorbalanced};
            \addplot[ultra thin, color=mpurple, mark=square, mark size=1.6pt,] table {\itrnextimrsvectorrelaxed};
        \end{groupplot}

    \end{tikzpicture}
    \end{adjustbox}

    \caption{Benchmark results of index sequentially and iterator.}
    \label{fig:index-sequentially}
\end{figure}

Unsurprisingly, the \stdvec{} shows the best results in this test. As it is backed by a contiguous chunk of memory, it takes full advantage of CPU cache locality. Besides, its structure is not affected by the method used to build it, when \rbtree{} and \rrbtree{} based vectors are.

Both balanced and unbalanced \imrsvec{} variants tend to be slower in comparison to \rrbvec{} in the \range{[100, \mega{1}]} input range by a factor of 2.06. For smaller inputs, \rrbvec{} it is slightly faster with a difference of 1.18.

\paragraph{Balanced vs. relaxed}
The difference between \rbvec{} and \rrbvec{} becomes noticeable as the problem size grows. The balanced variant is faster than the relaxed one by a factor of 2.68 in the \range{[100, \mega{1}]} input range. This is expected because \rrbvec{} introduces relaxed nodes, which rely on the size tables to compute the path to the value.

This, however, is not the case for small problem sizes in the \range{[0, 100]} range, for which the concatenation algorithm produces a balanced tree. Hence, both balanced and relaxed vectors demonstrate similar performance in that range.

\paragraph{Dynamic representation}
\pvec{} switches its internal representation from the standard vector to \rrbvec{} as soon as cloned. This is evident from the plot \ref{fig:index-sequentially}, where \pvec{} is 4.21 faster than \rbvec{}, but slower than \stdvec{} by a factor of 1.75.

\subsubsection*{Iterator}
Results of the iterator benchmarks show approximately ten-fold improvement in performance over sequential indexing. This is expected, as iterators read the contents of the tree by chunks rather than by index.

\stdvec{} shows the best results, with a difference of 1.98 on average compared to \pvec{}, and 9.12 in relation to \rrbvec{}. \imrsvec{} is 1.47 ahead of \rrbvec{} in the \range{[20, 100]} range.

\paragraph{Balanced vs. relaxed}
As iterator does not use size tables for index calculation for \rrbvec{}, it performs identically well compared to \rbvec{}. The same applies to \imrsvec{}.

\subsubsection*{Index randomly}
\begin{figure}[t]

    \center
    \begin{tikzpicture}
        \pgfplotstableread[col sep=comma]{data/index_randomly/im-rs-vector-balanced.csv}\imrsvectorbalanced;
        \pgfplotstableread[col sep=comma]{data/index_randomly/im-rs-vector-relaxed.csv}\imrsvectorrelaxed;
        \pgfplotstableread[col sep=comma]{data/index_randomly/pvec-std.csv}\pvecstd;
        \pgfplotstableread[col sep=comma]{data/index_randomly/pvec-rrbvec-balanced.csv}\pvecbalanced;
        \pgfplotstableread[col sep=comma]{data/index_randomly/pvec-rrbvec-relaxed.csv}\pvecrelaxed;
        \pgfplotstableread[col sep=comma]{data/index_randomly/rbvec.csv}\rbvecbalanced;
        \pgfplotstableread[col sep=comma]{data/index_randomly/rrbvec.csv}\rrbvecrelaxed;
        \pgfplotstableread[col sep=comma]{data/index_randomly/std-vec.csv}\stdvector;

        \begin{loglogaxis}[
            smooth,
            width=300pt,
            title={Index randomly},
            xlabel={Vector size (log scale)},
            ylabel={Mean time (log scale) [\millis{}]},
            ymajorgrids=true,
            xmajorgrids=true,
            grid style=dashed,
            legend pos=north west,
            legend style={draw=none,fill=none,font=\footnotesize},
            legend cell align=left,
            yticklabels={0, \micros{0.1}, \micros{1}, 0.01, 0.1, 1, 10, 100},
            xticklabels={0, 20, 100, \kilo{1}, \kilo{10}, \kilo{100}, \kilo{400}},
        ]
            \addplot[thin, color=morange, mark=*,] table {\stdvector};
            \addlegendentry{\stdvec{}}

            \addplot[thin, color=mred, mark=*,] table {\rbvecbalanced};
            \addlegendentry{\rbvec{}}

            \addplot[thin, color=mred, mark=pentagon,] table {\rrbvecrelaxed};
            \addlegendentry{\rrbvec{} \relaxed{}}

            \addplot[thin, color=mgreen, mark=*,] table {\pvecstd};
            \addlegendentry{\pvec{} \standard{}}

            \addplot[thin, color=mgreen, mark=square,] table {\pvecbalanced};
            \addlegendentry{\pvec{} \balanced{}}

            \addplot[thin, color=mgreen, mark=diamond*,] table {\pvecrelaxed};
            \addlegendentry{\pvec{} \relaxed{}}

            \addplot[thin, color=mpurple, mark=pentagon*,] table {\imrsvectorbalanced};
            \addlegendentry{\imrsvec{}}

            \addplot[thin, color=mpurple, mark=square,] table {\imrsvectorrelaxed};
            \addlegendentry{\imrsvec{} \relaxed{}}
        \end{loglogaxis}
    \end{tikzpicture}

    \caption{Benchmarking results of indexing randomly.}
    \label{fig:index-randomly}
\end{figure}

% > =================================================================================
In this benchmark, values will be read at random positions. Thus, it is quite likely that desired values will be located far apart in memory, potentially causing a cache invalidation. Additionally, results will show whether the performance degenerates with randomness, as it would with linked lists, for example.

By iterating \n{} times, a value is accessed at random index, which is generated within the \range{[0, N)} range by using the \crate{rand} crate\footnote{A Rust library for random number generation: \url{https://crates.io/crates/rand}}. According to the \crate{rand} documentation, generated indices are uniformly distributed. The number generator is explicitly seeded to produce the same stream of randomness between runs.
% < =================================================================================

A noticeable difference compared to the sequential benchmark is that the performance gap between \pvec{} and \stdvec{} is reduced from 1.76 to 1.14. The reason why \stdvec{} has lost its advantage is because of the frequent cache invalidation on access to random memory locations.

\paragraph{Balanced vs. relaxed}
\rbvec{} outperforms \rrbvec{} by 1.90 in the \range{[100, \mega{1}]} range. Both \rrbvec{} and \imrsvec{} are equally fast with insignificant marginal differences.

\paragraph{Dynamic representation}
It is clear that \pvec{}, when flat, is marginally slower compared to \stdvec{}. The performance difference remains consistent over the input range at 1.75.

\subsection{Updating}

% > =================================================================================
There are two dimensions in which the update operation will be evaluated. The first one, similar to the index operation, is the order in which vector values are updated: sequential and random. The second dimension introduces the clone operation used to reveal the cost of copying.

The benchmark list:
\begin{itemize}
    \item Sequentially, with and without \emph{clone}
    \item At random positions, with and without \emph{clone}
\end{itemize}

The setup routine for all benchmarks is identical. As for the index benchmarks, it generates both balanced and relaxed variants of the tree-based vectors. The type of inserted values is an unsigned 64-bit integer.

The vector size is determined by the benchmark argument. The problem size domain for tests using clone is \range{[20, \kilo{20}]}, which is smaller compared to the \range{[20, \kilo{100}]} range, used for benchmarks without clone. This is done to reduce the runtime of benchmarks.

\paragraph{The cost of naive clone vs. path copying}
One of the claimed advantages of \rbvec{} over \stdvec{}, is the cheap clone operation enabled by the path copying algorithm of \rbtree{}. \pvec{} takes advantage of that by switching from the flat to the tree-based representation when cloned. Hence, the objectives are:
\begin{itemize}
    \item Compare performance of naive and path copying algorithms.
    \item Evaluate the efficiency of dynamic representation in \pvec{}.
\end{itemize}

\paragraph{The overhead of relaxed nodes in \rrbtree{}}
Relaxed nodes of \rrbtree{} use size tables to keep track of the size of its child nodes. Balanced nodes, on the other hand, do not need them, as the size can be derived from the node level. Thus, relaxed nodes are more expensive to clone. Additionally, \rrbtree{} is not perfectly balanced as \rbtree{}, potentially resulting in taller trees. The results will reveal how significant this overhead is in practice.

\paragraph{Extending benchmarks with the clone operation}
The test with clone introduces an additional variable for keeping track of the cloned vector. This is done to ensure that at least two vector instances exist simultaneously when the update is executed. This is necessary because \rc{} pointers used to implement \rbtree{}, clone the underlying value on mutation only when the reference count is bigger than one. Thus, by having a cloned instance of vector present in the scope, we enforce the path copying algorithm to be used when updating a vector.
% < =================================================================================

Results of the update operation benchmarks are divided by sequential and random access, and complemented with evaluation with the use of clone operation.

\subsubsection*{Update sequentially}
\begin{figure}[!htbp]

    \center
    \begin{adjustbox}{width=\textwidth}
    \begin{tikzpicture}
        \tikzstyle{every node}=[
            font=\scriptsize,
            inner sep=2pt,
            outer sep=0pt
        ]

        \pgfplotstableread[col sep=comma]{data/update/im-rs-vector-balanced.csv}\upimrsvectorbalanced;
        \pgfplotstableread[col sep=comma]{data/update/im-rs-vector-relaxed.csv}\upimrsvectorrelaxed;
        \pgfplotstableread[col sep=comma]{data/update/pvec-std.csv}\uppvecstd;
        \pgfplotstableread[col sep=comma]{data/update/pvec-rrbvec-balanced.csv}\uppvecbalanced;
        \pgfplotstableread[col sep=comma]{data/update/pvec-rrbvec-relaxed.csv}\uppvecrelaxed;
        \pgfplotstableread[col sep=comma]{data/update/rbvec.csv}\uprbvecbalanced;
        \pgfplotstableread[col sep=comma]{data/update/rrbvec.csv}\uprrbvecrelaxed;
        \pgfplotstableread[col sep=comma]{data/update/std-vec.csv}\upstdvector;

        \pgfplotstableread[col sep=comma]{data/update_clone/im-rs-vector-balanced.csv}\upclimrsvectorbalanced;
        \pgfplotstableread[col sep=comma]{data/update_clone/im-rs-vector-relaxed.csv}\upclimrsvectorrelaxed;
        \pgfplotstableread[col sep=comma]{data/update_clone/pvec-std.csv}\upclpvecstd;
        \pgfplotstableread[col sep=comma]{data/update_clone/pvec-rrbvec-balanced.csv}\upclpvecbalanced;
        \pgfplotstableread[col sep=comma]{data/update_clone/pvec-rrbvec-relaxed.csv}\upclpvecrelaxed;
        \pgfplotstableread[col sep=comma]{data/update_clone/rbvec.csv}\upclrbvecbalanced;
        \pgfplotstableread[col sep=comma]{data/update_clone/rrbvec.csv}\upclrrbvecrelaxed;
        \pgfplotstableread[col sep=comma]{data/update_clone/std-vec.csv}\upclstdvector;

        \begin{groupplot}[
            group style={group size=2 by 1, horizontal sep=56pt,},
            xlabel={Vector size (log scale)},
            ylabel={Mean time (log scale) [\millis{}]},
            ymajorgrids=true,
            xmajorgrids=true,
            grid style=dashed,
        ]
            \nextgroupplot[
                xmode=log,
                ymode=log,
                title={Update sequentially},
                yticklabels={0, \micros{0.01}, \micros{0.1}, \micros{1}, 0.01, 0.1, 1, 10},
                xticklabels={0, 20, 100, \kilo{1}, \kilo{10}, \kilo{100}},
                legend columns=4,
                legend style={
                    at={(1.06,-0.2)},
                    anchor=north
                }
            ]
            \addplot[ultra thin, color=morange, mark=*, mark size=1.2pt,] table {\upstdvector};
            \addplot[ultra thin, color=mred, mark=*, mark size=1.2pt,] table {\uprbvecbalanced};
            \addplot[ultra thin, color=mred, mark=pentagon, mark size=1.6pt,] table {\uprrbvecrelaxed};
            \addplot[ultra thin, color=mgreen, mark=*, mark size=1.2pt,] table {\uppvecstd};
            \addplot[ultra thin, color=mgreen, mark=square, mark size=1.6pt,] table {\uppvecbalanced};
            \addplot[ultra thin, color=mgreen, mark=diamond*, mark size=1.2pt,] table {\uppvecrelaxed};
            \addplot[ultra thin, color=mpurple, mark=pentagon*, mark size=1.2pt,] table {\upimrsvectorbalanced};
            \addplot[ultra thin, color=mpurple, mark=square, mark size=1.6pt,] table {\upimrsvectorrelaxed};
            \legend{\stdvec{}, \rbvec{}, \rrbvec{} \relaxed{}, \pvec{} \standard{}, \pvec{} \balanced{}, \pvec{} \relaxed{}, \imrsvec{}, \imrsvec{} \relaxed{}}

            \nextgroupplot[
                xmode=log,
                ymode=log,
                title={Update sequentially and cloning},
                yticklabels={0, \micros{0.01}, \micros{0.1}, \micros{1}, 0.01, 0.1, 1, 10},
                xticklabels={0, 100, \kilo{1}, \kilo{10}},
            ]
            \addplot[ultra thin, color=morange, mark=*, mark size=1.2pt,] table {\upclstdvector};
            \addplot[ultra thin, color=mred, mark=*, mark size=1.2pt,] table {\upclrbvecbalanced};
            \addplot[ultra thin, color=mred, mark=pentagon, mark size=1.6pt,] table {\upclrrbvecrelaxed};
            \addplot[ultra thin, color=mgreen, mark=*, mark size=1.2pt,] table {\upclpvecstd};
            \addplot[ultra thin, color=mgreen, mark=square, mark size=1.6pt,] table {\upclpvecbalanced};
            \addplot[ultra thin, color=mgreen, mark=diamond*, mark size=1.2pt,] table {\upclpvecrelaxed};
            \addplot[ultra thin, color=mpurple, mark=pentagon*, mark size=1.2pt,] table {\upclimrsvectorbalanced};
            \addplot[ultra thin, color=mpurple, mark=square, mark size=1.6pt,] table {\upclimrsvectorrelaxed};
        \end{groupplot}
    \end{tikzpicture}
    \end{adjustbox}

    \caption{Benchmarking results of updating values sequentially.}
    \label{fig:update-sequentially}
\end{figure}

% > =================================================================================
The test function iterates over indices in the \range{[0, N)} range, where \n{} is the problem size, acquiring a mutable reference to the value at the given position. Once the reference is acquired, it is used to increment the value.
% < =================================================================================

\stdvec{} is the fastest vector in the sequential updates test, with the closest runner-up being \pvec{} with the mean difference of 2.32. When cloned, however, tree-based types, such as \rbvec{}, start outperforming the standard vector after surpassing the \kilo{4} size. The difference grows quickly, reaching 7.45 at the size of \kilo{20}. It demonstrates how well path copying scales when cloning large data structures.

Updates are slower for \imrsvec{} compared to \rrbvec{} by 2.47 on average. When cloned, however, the difference is less significant, varying mostly in the range of \range{[20, 400]}.

\paragraph{Balanced vs. relaxed}
Relaxed nodes and size tables associated with them were expected to have a negative impact on the performance when copying. Though, the numbers in the clone test do not confirm that assumption. However, if updated without cloning, \rrbvec{} is slower than \rbvec{} by 1.33.

\paragraph{Dynamic representation}
When updating a vector, \pvec{} is faster than both variants of \rbvec{} by a factor of 2.22 on average. It is, however, slower than \stdvec{}, even though the standard vector is used as a representation. It is expected that \pvec{} will introduce some overhead, as essentially, it introduces an additional abstraction layer.

\subsubsection*{Update randomly}
\begin{figure}[!htbp]

    \center
    \begin{adjustbox}{width=\textwidth}
    \begin{tikzpicture}
        \tikzstyle{every node}=[
            font=\scriptsize,
            inner sep=2pt,
            outer sep=0pt
        ]

        \pgfplotstableread[col sep=comma]{data/update_randomly/im-rs-vector-balanced.csv}\upimrsvectorbalanced;
        \pgfplotstableread[col sep=comma]{data/update_randomly/im-rs-vector-relaxed.csv}\upimrsvectorrelaxed;
        \pgfplotstableread[col sep=comma]{data/update_randomly/pvec-std.csv}\uppvecstd;
        \pgfplotstableread[col sep=comma]{data/update_randomly/pvec-rrbvec-balanced.csv}\uppvecbalanced;
        \pgfplotstableread[col sep=comma]{data/update_randomly/pvec-rrbvec-relaxed.csv}\uppvecrelaxed;
        \pgfplotstableread[col sep=comma]{data/update_randomly/rbvec.csv}\uprbvecbalanced;
        \pgfplotstableread[col sep=comma]{data/update_randomly/rrbvec.csv}\uprrbvecrelaxed;
        \pgfplotstableread[col sep=comma]{data/update_randomly/std-vec.csv}\upstdvector;

        \pgfplotstableread[col sep=comma]{data/update_clone_randomly/im-rs-vector-balanced.csv}\upclimrsvectorbalanced;
        \pgfplotstableread[col sep=comma]{data/update_clone_randomly/im-rs-vector-relaxed.csv}\upclimrsvectorrelaxed;
        \pgfplotstableread[col sep=comma]{data/update_clone_randomly/pvec-std.csv}\upclpvecstd;
        \pgfplotstableread[col sep=comma]{data/update_clone_randomly/pvec-rrbvec-balanced.csv}\upclpvecbalanced;
        \pgfplotstableread[col sep=comma]{data/update_clone_randomly/pvec-rrbvec-relaxed.csv}\upclpvecrelaxed;
        \pgfplotstableread[col sep=comma]{data/update_clone_randomly/rbvec.csv}\upclrbvecbalanced;
        \pgfplotstableread[col sep=comma]{data/update_clone_randomly/rrbvec.csv}\upclrrbvecrelaxed;
        \pgfplotstableread[col sep=comma]{data/update_clone_randomly/std-vec.csv}\upclstdvector;

        \begin{groupplot}[
            group style={group size=2 by 1, horizontal sep=56pt,},
            xlabel={Vector size (log scale)},
            ylabel={Mean time (log scale) [\millis{}]},
            yticklabels={0, \micros{0.1}, \micros{1}, 0.01, 0.1, 1, 10, 100},
            xticklabels={0, 20, 100, \kilo{1}, \kilo{10}, \kilo{100}},
            ymajorgrids=true,
            xmajorgrids=true,
            grid style=dashed,
        ]
            \nextgroupplot[
                xmode=log,
                ymode=log,
                title={Update randomly},
                legend columns=4,
                legend style={
                    at={(1.06,-0.2)},
                    anchor=north
                }
            ]
            \addplot[ultra thin, color=morange, mark=*, mark size=1.2pt,] table {\upstdvector};
            \addplot[ultra thin, color=mred, mark=*, mark size=1.2pt,] table {\uprbvecbalanced};
            \addplot[ultra thin, color=mred, mark=pentagon, mark size=1.6pt,] table {\uprrbvecrelaxed};
            \addplot[ultra thin, color=mgreen, mark=*, mark size=1.2pt,] table {\uppvecstd};
            \addplot[ultra thin, color=mgreen, mark=square, mark size=1.6pt,] table {\uppvecbalanced};
            \addplot[ultra thin, color=mgreen, mark=diamond*, mark size=1.2pt,] table {\uppvecrelaxed};
            \addplot[ultra thin, color=mpurple, mark=pentagon*, mark size=1.2pt,] table {\upimrsvectorbalanced};
            \addplot[ultra thin, color=mpurple, mark=square, mark size=1.6pt,] table {\upimrsvectorrelaxed};
            \legend{\stdvec{}, \rbvec{}, \rrbvec{} \relaxed{}, \pvec{} \standard{}, \pvec{} \balanced{}, \pvec{} \relaxed{}, \imrsvec{}, \imrsvec{} \relaxed{}}

            \nextgroupplot[
                xmode=log,
                ymode=log,
                title={Update randomly and cloning},
                yticklabels={0, \micros{0.1}, \micros{1}, 0.01, 0.1, 1, 10, 100},
                xticklabels={0, 100, \kilo{1}, \kilo{10}},
            ]
            \addplot[ultra thin, color=morange, mark=*, mark size=1.2pt,] table {\upclstdvector};
            \addplot[ultra thin, color=mred, mark=*, mark size=1.2pt,] table {\upclrbvecbalanced};
            \addplot[ultra thin, color=mred, mark=pentagon, mark size=1.6pt,] table {\upclrrbvecrelaxed};
            \addplot[ultra thin, color=mgreen, mark=*, mark size=1.2pt,] table {\upclpvecstd};
            \addplot[ultra thin, color=mgreen, mark=square, mark size=1.6pt,] table {\upclpvecbalanced};
            \addplot[ultra thin, color=mgreen, mark=diamond*, mark size=1.2pt,] table {\upclpvecrelaxed};
            \addplot[ultra thin, color=mpurple, mark=pentagon*, mark size=1.2pt,] table {\upclimrsvectorbalanced};
            \addplot[ultra thin, color=mpurple, mark=square, mark size=1.6pt,] table {\upclimrsvectorrelaxed};
        \end{groupplot}
    \end{tikzpicture}
    \end{adjustbox}

    \caption{Benchmarking results of updating values randomly.}
    \label{fig:update-randomly}
\end{figure}

% > =================================================================================
The test function contains a loop, which is executed \n{} times. In the loop body, value is updated by incrementing it, at the index that is randomly generated in the \range{[0, N)} range.
% < =================================================================================

As with indexing, when values are updated randomly rather than sequentially, the performance gap between the \stdvec{} and \pvec{} became as small as 1.11 due to the frequent cache invalidations caused by random access. Other than that, there are no significant distinctions.

\subsection{Pushing}
% > =================================================================================
The push operation is evaluated by populating an empty and existing vectors. Both tests are also extended with the clone operation.

\paragraph{The overhead of relaxed nodes in \rrbtree{}}
The push operation is responsible for increasing the vector capacity. While the vector capacity calculation for \rrbtree{} relies on the size tables, for \rbtree{}, it is sufficient to know the level of the node and the branching factor. Additionally, instantiating relaxed nodes implies the allocation of size tables. All these factors combined are expected to make \rrbtree{}'s push slower compared to \rbtree{}. Thus, there is a dedicated benchmark that uses prebuilt, \rrbtree{}, and \rbtree{} based vectors to evaluate the difference.

\paragraph{Extending benchmarks with the clone operation}
To force \pvec{} to switch from \stdvec{} to \rrbvec{}, several preconditions have to be met, including the reference count being bigger than 1. Hence, the benchmarks above are extended to use clone, in the same way as for benchmarks of the update operation. Beyond the evaluation of \pvec{}, the results are expected to reveal how tree-based vectors stack up to \stdvec{}. The input range for benchmarks using clone is \range{[20, \kilo{40}]}.
% < =================================================================================

Benchmarks are divided into two use cases:
\begin{itemize}
    \item Building a new vector from scratch by pushing values into it.
    \item Pushing values into an existing vector, both balanced and relaxed.
\end{itemize}

Both benchmarks are extended using the clone operation.

\subsubsection*{Building a vector}
\begin{figure}[!htbp]

    \center
    \begin{adjustbox}{width=\textwidth}
    \begin{tikzpicture}
        \tikzstyle{every node}=[
            font=\scriptsize,
            inner sep=2pt,
            outer sep=0pt
        ]

        \pgfplotstableread[col sep=comma]{data/push/im-rs-vector-balanced.csv}\pushimrsvectorbalanced;
        \pgfplotstableread[col sep=comma]{data/push/pvec-std.csv}\pushpvecstd;
        \pgfplotstableread[col sep=comma]{data/push/rbvec.csv}\pushrbvecbalanced;
        \pgfplotstableread[col sep=comma]{data/push/std-vec.csv}\pushstdvector;

        \pgfplotstableread[col sep=comma]{data/push_clone/im-rs-vector-balanced.csv}\pushclnimrsvectorbalanced;
        \pgfplotstableread[col sep=comma]{data/push_clone/pvec-rrbvec-balanced.csv}\pushclnpvecbalanced;
        \pgfplotstableread[col sep=comma]{data/push_clone/rbvec.csv}\pushclnrbvecbalanced;
        \pgfplotstableread[col sep=comma]{data/push_clone/std-vec.csv}\pushclnstdvector;

        \begin{groupplot}[
            group style={group size=2 by 1, horizontal sep=56pt,},
            xlabel={Vector size (log scale)},
            ylabel={Mean time (log scale) [\millis{}]},
            xticklabels={0, 10, 100, \kilo{1}, \kilo{10}, \kilo{100}, \mega{1}},
            yticklabels={0, \micros{0.01}, \micros{0.1}, \micros{1}, 0.01, 0.1, 1, 10},
            ymajorgrids=true,
            xmajorgrids=true,
            grid style=dashed,
        ]
            \nextgroupplot[
                xmode=log,
                ymode=log,
                title={Building a new vector},
                legend columns=4,
                legend style={
                    at={(0.5,-0.2)},
                    anchor=north
                }
            ]
            \addplot[ultra thin, color=morange, mark=*, mark size=1.2pt,] table {\pushstdvector};
            \addplot[ultra thin, color=mred, mark=*, mark size=1.2pt,] table {\pushrbvecbalanced};
            \addplot[ultra thin, color=mgreen, mark=*, mark size=1.2pt,] table {\pushpvecstd};
            \addplot[ultra thin, color=mpurple, mark=pentagon*, mark size=1.2pt,] table {\pushimrsvectorbalanced};
            \legend{\stdvec{}, \rbvec{}, \pvec{} \standard{}, \imrsvec{}}

            \nextgroupplot[
                xmode=log,
                ymode=log,
                title={Building a new vector and cloning it.},
                xticklabels={0, 20, 100, \kilo{1}, \kilo{10}},
                yticklabels={0, \micros{0.1}, \micros{1}, 0.01, 0.1, 1, 10},
                legend columns=4,
                legend style={
                    at={(0.5,-0.2)},
                    anchor=north
                }
            ]
            \addplot[ultra thin, color=morange, mark=*, mark size=1.2pt,] table {\pushclnstdvector};
            \addplot[ultra thin, color=mred, mark=*, mark size=1.2pt,] table {\pushclnrbvecbalanced};
            \addplot[ultra thin, color=mgreen, mark=square, mark size=1.6pt,] table {\pushclnpvecbalanced};
            \addplot[ultra thin, color=mpurple, mark=pentagon*, mark size=1.2pt,] table {\pushclnimrsvectorbalanced};
            \legend{\stdvec{}, \rbvec{}, \pvec{}, \imrsvec{}}
        \end{groupplot}
    \end{tikzpicture}
    \end{adjustbox}

    \caption{Benchmarking results of push.}
    \label{fig:push}
\end{figure}

% > =================================================================================
As vector is built from scratch in this benchmark, there is no need for a setup routine. The test function runs a loop over the \range{[0, N)} range of indices, and pushes the index as a value into a vector. The problem size range is \range{[20, \mega{1}]}.
% < =================================================================================

When building a vector in the \range{[20, 100]} range, tree-based types are slightly faster than \stdvec{}, as they take advantage of the tail optimization.

\imrsvec{} is almost as fast as \rbvec{} with a difference of 1.18. When cloned, however, \rbvec{} outperforms it by a significant amount of 3.61.

In the cloning benchmark, persistent vectors once more demonstrate how effective they are when copied, by \rbvec{} being faster than \stdvec{} by a staggering factor of 12.85.

\paragraph{Dynamic representation}
Overall, \pvec{} is the fastest vector, mostly due to the combination of using \stdvec{} with the pre-allocated space\footnote{\pvec{} is initialized to the capacity of branching factor, that is 32 in the test configuration.} for small sizes. Closer to the end of the size range, \stdvec{} and \pvec{} align in performance with the average difference of 1.27.

When cloned, \pvec{} is losing only to \rbvec{} by in-significant 1.35, demonstrating the effectiveness of the dynamic representation.

\subsubsection*{Adding values to an existing vector}
\begin{figure}[!htbp]

    \center
    \begin{adjustbox}{width=\textwidth}
    \begin{tikzpicture}
        \tikzstyle{every node}=[
            font=\scriptsize,
            inner sep=2pt,
            outer sep=0pt
        ]

        \pgfplotstableread[col sep=comma]{data/push_relaxed/im-rs-vector-balanced.csv}\pushimrsvectorbalanced;
        \pgfplotstableread[col sep=comma]{data/push_relaxed/im-rs-vector-relaxed.csv}\pushimrsvectorrelaxed;
        \pgfplotstableread[col sep=comma]{data/push_relaxed/pvec-std.csv}\pushpvecstd;
        \pgfplotstableread[col sep=comma]{data/push_relaxed/pvec-rrbvec-balanced.csv}\pushpvecbalanced;
        \pgfplotstableread[col sep=comma]{data/push_relaxed/pvec-rrbvec-relaxed.csv}\pushpvecrelaxed;
        \pgfplotstableread[col sep=comma]{data/push_relaxed/rbvec.csv}\pushrbvecbalanced;
        \pgfplotstableread[col sep=comma]{data/push_relaxed/rrbvec.csv}\pushrrbvecrelaxed;
        \pgfplotstableread[col sep=comma]{data/push_relaxed/std-vec.csv}\pushstdvector;

        \pgfplotstableread[col sep=comma]{data/push_clone_relaxed/im-rs-vector-balanced.csv}\pushclnimrsvectorbalanced;
        \pgfplotstableread[col sep=comma]{data/push_clone_relaxed/im-rs-vector-relaxed.csv}\pushclnimrsvectorrelaxed;
        \pgfplotstableread[col sep=comma]{data/push_clone_relaxed/pvec-rrbvec-balanced.csv}\pushclnpvecbalanced;
        \pgfplotstableread[col sep=comma]{data/push_clone_relaxed/pvec-rrbvec-relaxed.csv}\pushclnpvecrelaxed;
        \pgfplotstableread[col sep=comma]{data/push_clone_relaxed/rbvec.csv}\pushclnrbvecbalanced;
        \pgfplotstableread[col sep=comma]{data/push_clone_relaxed/rrbvec.csv}\pushclnrrbvecrelaxed;
        \pgfplotstableread[col sep=comma]{data/push_clone_relaxed/std-vec.csv}\pushclnstdvector;

        \begin{groupplot}[
            group style={group size=2 by 1, horizontal sep=56pt,},
            xlabel={Vector size (log scale)},
            ylabel={Mean time (log scale) [\millis{}]},
            xticklabels={0, 20, 100, \kilo{1}, \kilo{10}, \kilo{100}},
            yticklabels={0, \micros{0.1}, \micros{1}, 0.01, 0.1, 1},
            ymajorgrids=true,
            xmajorgrids=true,
            grid style=dashed,
        ]
            \nextgroupplot[
                xmode=log,
                ymode=log,
                title={Adding values to an existing vector},
                legend columns=4,
                legend style={
                    at={(0.5,-0.2)},
                    anchor=north
                }
            ]
            \addplot[ultra thin, color=morange, mark=*, mark size=1.2pt,] table {\pushstdvector};
            \addplot[ultra thin, color=mred, mark=*, mark size=1.2pt,] table {\pushrbvecbalanced};
            \addplot[ultra thin, color=mred, mark=pentagon, mark size=1.6pt,] table {\pushrrbvecrelaxed};
            \addplot[ultra thin, color=mgreen, mark=*, mark size=1.2pt,] table {\pushpvecstd};
            \addplot[ultra thin, color=mgreen, mark=square, mark size=1.6pt,] table {\pushpvecbalanced};
            \addplot[ultra thin, color=mgreen, mark=diamond*, mark size=1.2pt,] table {\pushpvecrelaxed};
            \addplot[ultra thin, color=mpurple, mark=pentagon*, mark size=1.2pt,] table {\pushimrsvectorbalanced};
            \addplot[ultra thin, color=mpurple, mark=square, mark size=1.6pt,] table {\pushimrsvectorrelaxed};
            \legend{\stdvec{}, \rbvec{}, \rrbvec{} \relaxed{}, \pvec{} \standard{}, \pvec{}, \pvec{} \relaxed{}, \imrsvec{}, \imrsvec{} \relaxed{}}

            \nextgroupplot[
                xmode=log,
                ymode=log,
                xticklabels={0, 20, 100, \kilo{1}, \kilo{10}, \kilo{20}},
                yticklabels={0, \micros{0.1}, \micros{1}, 0.01, 0.1, 1, 10, 100},
                title={Adding values to an existing vector and cloning it},
                legend columns=4,
                legend style={
                    at={(0.5,-0.2)},
                    anchor=north
                }
            ]
            \addplot[ultra thin, color=morange, mark=*, mark size=1.2pt,] table {\pushclnstdvector};
            \addplot[ultra thin, color=mred, mark=*, mark size=1.2pt,] table {\pushclnrbvecbalanced};
            \addplot[ultra thin, color=mred, mark=pentagon, mark size=1.6pt,] table {\pushclnrrbvecrelaxed};
            \addplot[ultra thin, color=mgreen, mark=square, mark size=1.6pt,] table {\pushclnpvecbalanced};
            \addplot[ultra thin, color=mgreen, mark=diamond*, mark size=1.2pt,] table {\pushclnpvecrelaxed};
            \addplot[ultra thin, color=mpurple, mark=pentagon*, mark size=1.2pt,] table {\pushclnimrsvectorbalanced};
            \addplot[ultra thin, color=mpurple, mark=square, mark size=1.6pt,] table {\pushclnimrsvectorrelaxed};
            \legend{\stdvec{}, \rbvec{}, \rrbvec{} \relaxed{}, \pvec{}, \pvec{} \relaxed{}, \imrsvec{}, \imrsvec{} \relaxed{}}
        \end{groupplot}
    \end{tikzpicture}
    \end{adjustbox}

    \caption{Benchmarking results of adding values to an existing vector.}
    \label{fig:push-existing}
\end{figure}

% > =================================================================================
Push operation does not produce relaxed nodes in balanced trees. Hence, there is no way to evaluate the impact of relaxation in the benchmark of building a vector from scratch. Thus, in this benchmark, values are added to an existing vector, where vector can be both balanced and relaxed depending on the setup routine.

Objectives of the benchmark are:
\begin{itemize}
    \item Check the cost of the complex sub-tree capacity and index computation.
    \item Measure the overhead of using size tables when cloning relaxed nodes.
\end{itemize}

The setup routine generates a vector of the fixed size of \n{} and passes it to the test function. The balanced, \rbtree{} based vector is created by pushing values directly into it, while the \rrbtree{} based one, is created by concatenating several vectors together. Once a vector is created, the test function pushes \n{} values onto it.
% < =================================================================================

Even though size tables increase the size of the node and add complexity to the implementation, both relaxed and balanced trees show nearly identical performance in this test.

When push is called continuously, even for relaxed nodes, only balanced nodes are added to the tree. Eventually, all new nodes at the end of the tree, except the root, will be balanced. Thus, there is nearly no overhead of running push over a relaxed \rrbvec{} in the given benchmarks.

\subsection{Popping}

\begin{figure}[!htbp]

    \center
    \begin{adjustbox}{width=\textwidth}
    \begin{tikzpicture}
        \tikzstyle{every node}=[
            font=\scriptsize,
            inner sep=2pt,
            outer sep=0pt
        ]

        \pgfplotstableread[col sep=comma]{data/pop/im-rs-vector-balanced.csv}\popimrsvectorbalanced;
        \pgfplotstableread[col sep=comma]{data/pop/im-rs-vector-relaxed.csv}\popimrsvectorrelaxed;
        \pgfplotstableread[col sep=comma]{data/pop/pvec-std.csv}\poppvecstd;
        \pgfplotstableread[col sep=comma]{data/pop/pvec-rrbvec-balanced.csv}\poppvecbalanced;
        \pgfplotstableread[col sep=comma]{data/pop/pvec-rrbvec-relaxed.csv}\poppvecrelaxed;
        \pgfplotstableread[col sep=comma]{data/pop/rbvec.csv}\poprbvecbalanced;
        \pgfplotstableread[col sep=comma]{data/pop/rrbvec.csv}\poprrbvecrelaxed;
        \pgfplotstableread[col sep=comma]{data/pop/std-vec.csv}\popstdvector;

        \pgfplotstableread[col sep=comma]{data/pop_clone/im-rs-vector-balanced.csv}\popclnimrsvectorbalanced;
        \pgfplotstableread[col sep=comma]{data/pop_clone/im-rs-vector-relaxed.csv}\popclnimrsvectorrelaxed;
        \pgfplotstableread[col sep=comma]{data/pop_clone/pvec-std.csv}\popclnpvecstd;
        \pgfplotstableread[col sep=comma]{data/pop_clone/pvec-rrbvec-balanced.csv}\popclnpvecbalanced;
        \pgfplotstableread[col sep=comma]{data/pop_clone/pvec-rrbvec-relaxed.csv}\popclnpvecrelaxed;
        \pgfplotstableread[col sep=comma]{data/pop_clone/rbvec.csv}\popclnrbvecbalanced;
        \pgfplotstableread[col sep=comma]{data/pop_clone/rrbvec.csv}\popclnrrbvecrelaxed;
        \pgfplotstableread[col sep=comma]{data/pop_clone/std-vec.csv}\popclnstdvector;

        \begin{groupplot}[
            group style={group size=2 by 1, horizontal sep=56pt,},
            xlabel={Vector size (log scale)},
            ylabel={Mean time (log scale) [\millis{}]},
            xticklabels={0, 20, 100, \kilo{1}, \kilo{10}, \kilo{60}},
            yticklabels={0, \micros{0.1}, \micros{1}, 0.01, 0.1, 1},
            ymajorgrids=true,
            xmajorgrids=true,
            grid style=dashed,
        ]
            \nextgroupplot[
                xmode=log,
                ymode=log,
                title={Popping values from a vector},
                legend columns=4,
                legend style={
                    at={(1.12,-0.2)},
                    anchor=north
                }
            ]
            \addplot[ultra thin, color=morange, mark=*, mark size=1.2pt,] table {\popstdvector};
            \addplot[ultra thin, color=mred, mark=*, mark size=1.2pt,] table {\poprbvecbalanced};
            \addplot[ultra thin, color=mred, mark=pentagon, mark size=1.6pt,] table {\poprrbvecrelaxed};
            \addplot[ultra thin, color=mgreen, mark=*, mark size=1.2pt,] table {\poppvecstd};
            \addplot[ultra thin, color=mgreen, mark=square, mark size=1.6pt,] table {\poppvecbalanced};
            \addplot[ultra thin, color=mgreen, mark=diamond*, mark size=1.2pt,] table {\poppvecrelaxed};
            \addplot[ultra thin, color=mpurple, mark=pentagon*, mark size=1.2pt,] table {\popimrsvectorbalanced};
            \addplot[ultra thin, color=mpurple, mark=square, mark size=1.6pt,] table {\popimrsvectorrelaxed};
            \legend{\stdvec{}, \rbvec{}, \rrbvec{} \relaxed{}, \pvec{} \standard{}, \pvec{}, \pvec{} \relaxed{}, \imrsvec{}, \imrsvec{} \relaxed{}}

            \nextgroupplot[
                xmode=log,
                ymode=log,
                xticklabels={0, 20, 100, \kilo{1}, \kilo{10}},
                yticklabels={0, \micros{1}, 0.01, 0.1, 1, 10, 100},
                title={Popping values from a vector and cloning it},
            ]
            \addplot[ultra thin, color=morange, mark=*, mark size=1.2pt,] table {\popclnstdvector};
            \addplot[ultra thin, color=mred, mark=*, mark size=1.2pt,] table {\popclnrbvecbalanced};
            \addplot[ultra thin, color=mred, mark=pentagon, mark size=1.6pt,] table {\popclnrrbvecrelaxed};
            \addplot[ultra thin, color=mgreen, mark=*, mark size=1.2pt,] table {\popclnpvecstd};
            \addplot[ultra thin, color=mgreen, mark=square, mark size=1.6pt,] table {\popclnpvecbalanced};
            \addplot[ultra thin, color=mgreen, mark=diamond*, mark size=1.2pt,] table {\popclnpvecrelaxed};
            \addplot[ultra thin, color=mpurple, mark=pentagon*, mark size=1.2pt,] table {\popclnimrsvectorbalanced};
            \addplot[ultra thin, color=mpurple, mark=square, mark size=1.6pt,] table {\popclnimrsvectorrelaxed};
        \end{groupplot}
    \end{tikzpicture}
    \end{adjustbox}

    \caption{Benchmarking results of popping values.}
    \label{fig:pop}
\end{figure}
% > =================================================================================
The pop operation manages the vector capacity as well as push. For \stdvec{}, it means shrinking the array and copying elements over. For tree-based vectors, it implies de-allocating nodes and reducing the height of the tree when necessary.

The benchmark is divided into two tests, namely pop and pop clone. The first test calls pop continuously in the loop until the vector is emptied, with the problem size range of \range{[20, \kilo{60}]}.

In the second benchmark, each pop operation will be followed by a clone. Both tests include balanced and relaxed vector types, which are prepared in the setup routine. The problem size range is \range{[20, \kilo{40}]}.

\paragraph{The overhead of relaxed nodes in \rrbtree{}}
Lowering the height of \rrbtree{} involves the tree capacity calculation using size tables, that comes at an additional cost. Thus, this test includes both balanced and relaxed variants of vectors.
% < =================================================================================

In the test without clone, \imrsvec{} is slightly faster than \rbvec{} and \rrbvec{} by 1.29. The difference between balanced and relaxed trees is only 1.02, showing that the cost of managing the relaxed tree capacity is not expensive.

\pvec{} is the second fastest vector with \stdvec{} being 1.99 faster on average. When cloned, \pvec{} switches the representation and becomes as fast as \rbvec{}.

On the other hand, \imrsvec{} is the slowest tree-based vector when cloned, but still faster than \stdvec{}, especially for large sizes.

\subsection{Appending}

\begin{figure}[t]

    \center
    \begin{tikzpicture}
        \pgfplotstableread[col sep=comma]{data/append/im-rs-vector-relaxed.csv}\imrsvectorrelaxed;
        \pgfplotstableread[col sep=comma]{data/append/pvec-std.csv}\pvecstd;
        \pgfplotstableread[col sep=comma]{data/append/pvec-rrbvec-relaxed.csv}\pvecrelaxed;
        \pgfplotstableread[col sep=comma]{data/append/rbvec.csv}\rbvecbalanced;
        \pgfplotstableread[col sep=comma]{data/append/rrbvec.csv}\rrbvecrelaxed;
        \pgfplotstableread[col sep=comma]{data/append/std-vec.csv}\stdvector;

        \begin{loglogaxis}[
            smooth,
            width=300pt,
            title={Appending vectors},
            xlabel={Vector size (log scale)},
            ylabel={Mean time (log scale) [\millis{}]},
            xticklabels={0, 10, 100, \kilo{1}, \kilo{10}, \kilo{100}, \mega{1}},
            yticklabels={0, \micros{1}, 0.01, 0.1, 1, 10},
            ymajorgrids=true,
            xmajorgrids=true,
            grid style=dashed,
            legend pos=north west,
            legend style={draw=none,fill=none,font=\footnotesize},
            legend cell align=left,
        ]
            \addplot[thin, color=morange, mark=*,] table {\stdvector};
            \addlegendentry{\stdvec{}}

            \addplot[thin, color=mred, mark=*,] table {\rbvecbalanced};
            \addlegendentry{\rbvec{}}

            \addplot[thin, color=mred, mark=pentagon,] table {\rrbvecrelaxed};
            \addlegendentry{\rrbvec{} \relaxed{}}

            \addplot[thin, color=mgreen, mark=*,] table {\pvecstd};
            \addlegendentry{\pvec{} \standard{}}

            \addplot[thin, color=mgreen, mark=diamond*,] table {\pvecrelaxed};
            \addlegendentry{\pvec{} \relaxed{}}

            \addplot[thin, color=mpurple, mark=square,] table {\imrsvectorrelaxed};
            \addlegendentry{\imrsvec{} \relaxed{}}
        \end{loglogaxis}
    \end{tikzpicture}

    \caption{Benchmarking results of appending vectors.}
    \label{fig:append}
\end{figure}

% > =================================================================================
The append operation merges contents of one vector into another. One of the advantages of \rrbtree{} is the relatively low cost of append, that is \bigo{(m^2 \cdot log_m(n)}, in comparison to \bigo{max(a,b)} of \stdvec{}. The objective is to confirm this assumption experimentally.

\paragraph{Naive vs. relaxed append algorithm}
The \rbtree{}-based vector uses a naive concatenation algorithm that moves values from one vector to another. \rrbtree{}, on the other hand, merges and re-balances two trees, which is faster in theory. Due to the hardware design specifics, this might not be true for all vector sizes. Thus, benchmarks will reveal how different algorithms perform depending on the size of concatenated vectors.

\paragraph{Appending vectors}
The setup routine prepares a collection of vectors, where each consecutive vector is bigger than the previous. The total size of all prepared vectors adds up to the problem size \n{}. The benchmark is parameterized over the vector size, which will be in the \range{[20, \mega{1}]} range.

Each vector is created by a combination of append and push operations. This way \pvec{} and \rrbvec{} will be forced to use \rrbtree{} for internal representation, while \rbvec{} will remain balanced. \stdvec{} remains flat and does not depend on the type of operation used to add values to it.

The benchmark function iterates over generated vectors and appends them into a vector defined as a local variable.
% < =================================================================================

Based on the results, the naive copying of \stdvec{} is the fastest concatenation algorithm up to the problem size of \kilo{40}. However, after surpassing \kilo{40} it quickly degenerates and loses to \rrbvec{} by 6.53, with a maximum difference of 12.85 for the size of \mega{1}.

Even though not for all input sizes, we can see that \rrbvec{}'s concatenation algorithm scales better and eventually outperforms \stdvec{}.

\imrsvec{} catches up to \stdvec{} only at the size of \kilo{400}, still being slower than \rrbvec{} by 1.29 at that point.

\paragraph{Balanced vs. relaxed}
For the input size up to \kilo{2}, \rbvec{} and \rrbvec{} show similar results. The simplicity of naive concatenation used by \rbvec{} is sufficient to be as fast as \rrbvec{} due to the small problem size.

This, however, drastically changes after the \kilo{2} size, where \rbvec{} continuously degrades, showing the worst results among all vectors. The performance difference between \rbvec{} and \rrbvec{} at the size of \mega{1} is staggering 145.59.

As for \pvec{}, it follows the curves of \stdvec{} and \rrbvec{} because of the dynamic representation, as expected.

\subsection{Splitting}

\begin{figure}[t]

    \center
    \begin{tikzpicture}
        \pgfplotstableread[col sep=comma]{data/split_off/im-rs-vector-relaxed.csv}\imrsvectorrelaxed;
        \pgfplotstableread[col sep=comma]{data/split_off/pvec-std.csv}\pvecstd;
        \pgfplotstableread[col sep=comma]{data/split_off/pvec-rrbvec-relaxed.csv}\pvecrelaxed;
        \pgfplotstableread[col sep=comma]{data/split_off/rbvec.csv}\rbvecbalanced;
        \pgfplotstableread[col sep=comma]{data/split_off/rrbvec.csv}\rrbvecrelaxed;
        \pgfplotstableread[col sep=comma]{data/split_off/std-vec.csv}\stdvector;

        \begin{loglogaxis}[
            smooth,
            width=300pt,
            title={Splitting vectors},
            xlabel={Vector size (log scale)},
            ylabel={Mean time (log scale) [\millis{}]},
            xticklabels={0, 100, \kilo{1}, \kilo{10}, \kilo{100}, \kilo{400}},
            yticklabels={0, \micros{0.1}, \micros{1}, 0.01, 0.1, 1, 10, 100, \seconds{1}, \seconds{10}},
            ymajorgrids=true,
            xmajorgrids=true,
            grid style=dashed,
            legend pos=north west,
            legend style={draw=none,fill=none,font=\footnotesize},
            legend cell align=left,
        ]
            \addplot[thin, color=morange, mark=*,] table {\stdvector};
            \addlegendentry{\stdvec{}}

            \addplot[thin, color=mred, mark=*,] table {\rbvecbalanced};
            \addlegendentry{\rbvec{}}

            \addplot[thin, color=mred, mark=pentagon,] table {\rrbvecrelaxed};
            \addlegendentry{\rrbvec{} \relaxed{}}

            \addplot[thin, color=mgreen, mark=*,] table {\pvecstd};
            \addlegendentry{\pvec{} \standard{}}

            \addplot[thin, color=mgreen, mark=diamond*,] table {\pvecrelaxed};
            \addlegendentry{\pvec{} \relaxed{}}

            \addplot[thin, color=mpurple, mark=square,] table {\imrsvectorrelaxed};
            \addlegendentry{\imrsvec{} \relaxed{}}
        \end{loglogaxis}
    \end{tikzpicture}

    \caption{Benchmarking results of splitting vectors.}
    \label{fig:split}
\end{figure}

% > =================================================================================
The split operation slices a vector into two parts at the given index. The \rrbtree{}'s algorithm theoretically can achieve good performance by avoiding unnecessary copying. However, due to its complexity, it might be outperformed by naive copying for small-sized vectors.

\paragraph{Splitting vectors}
The test itself anticipates a prepared vector, which is generated in the setup routine. To evaluate both balanced and relaxed variants, it generates them either by using appending or pushing. The vector sizes are within \range{[128, \kilo{200}]}.

Once a vector is generated, the benchmark function enters the loop with the condition that the vector needs to contain more than 64 elements. In the loop, a vector is split at index 64, the result of which is assigned back to a variable. Essentially, a vector is being truncated at the front by 64 elements, until it is small enough for the loop to exit.
% < =================================================================================


% TODO: you have to be open about:
%  1: split_off and append are fast*er* compared to vec only for large sizes
%   1.1: this is part of the reason why in par benchmarks tree-based vectors suck
%  2: split_off benchmark is biased as it favors the use-case that is faster for the tree-based implementations

The performance advantage of \stdvec{} over \rrbvec{} is 3.24 up to the size of \kilo{20}, after which it degrades and gets slower than \imrsvec{} and \rrbvec{} by 5.95. The difference is more significant for \rbvec{} at the factor of 65.65 for the size of \kilo{400}.

% NOTE: this can go under the summary part of the 
%  section, rather than being its own sub-section
\subsection{Rc vs Arc}
\label{sec:perf-rc-vs-arc}
\todo{Execute sequential benchmarks using \arc{} pointer}
\stdvec{} is not included in the comparison, as its implementation does not rely on the reference counted pointers.

\section{Parallel benchmark results}
\label{sec:par-benchmarks}

% > =================================================================================
\subsection*{Computation stages}
The parallel benchmarks for this project are executed in three stages:
\begin{enumerate}
    \item Split the work into tasks between threads.
    \item Process the tasks.
    \item Combine and return the results.
\end{enumerate}

% TODO: you haven't really described how results are collected, and what are Fold and Reduce combinators.
% TODO: throwing a snippet in won't help.
The third step can be either of two:
\begin{itemize}
    \item Collect individual items into a vector using the parallel \emph{Fold} combinator.
    \item Reduce emitted vectors into a single one using the \emph{Reduce} combinator.
\end{itemize}

\begin{figure}[!htbp]
    \centering

    \begin{minted}{rust}
        let result = parallel_iterator
            .fold(Vec::new, |mut vec, x| {
                vec.push(x);
                vec
            })
            .reduce(Vec::new, |mut vec1, mut vec2| {
                vec1.append(&mut vec2);
                vec1
            });
    \end{minted}

    \caption{Collecting items of parallel iterator.}
    \label{fig:fold-reduce}
\end{figure}
% < =================================================================================

% > =================================================================================
The benchmarks were executed against \stdvec{} and vectors from \pvecrs{}, where \rbvec{}, \rrbvec{}, and \pvec{} were compiled with the threadsafe reference-counted pointer -- \arc{}.

As of the time of writing, \imrsvec{} does not implement the Rayon's \emph{IntoParallelIterator} trait. Since all other tested vectors rely on it, the decision was made not to include \imrsvec{} in the parallel benchmarks to avoid the unfair comparison.

Benchmarks were parameterized over two dimensions: the vector size and the number of threads. To see whether parallelism is beneficial, each benchmark has an analogous, sequential implementation used as the baseline.
% < =================================================================================

The results present the overall vector performance across several operations in the parallel context to check the following:
\begin{itemize}
    \item Scalability of the tree-based vectors.
    \item The impact of the relaxed append and split operations.
\end{itemize}

The results are presented in the form of a three-dimensional graph, where x and y-axis correspond to the problem size and number of used threads, while z stands for the mean runtime.

\subsection{Adding elements of two vectors.}

\begin{figure}[!htbp]

    \center
    \begin{tikzpicture}
        \tikzstyle{every node}=[
            font=\footnotesize,
            inner sep=2pt,
            outer sep=0pt
        ]

        \begin{loglogaxis}[
            smooth,
            width=300pt,
            title style={align=center},
            title={Adding elements of two \n{} sized vectors\\ parallelized on \emph{K} number of threads.},
            ymajorgrids=true,
            xmajorgrids=true,
            zmajorgrids=true,
            xlabel={Vector size (log scale)},
            ylabel={Number of threads (log scale)},
            zlabel={Mean time (log scale) [\millis{}]},
            xticklabels={10, 100, \kilo{1}, \kilo{10}},
            ytick={1, 2, 4, 8},
            yticklabels={1, 2, 4, 8},
            zticklabel style={
                /pgf/number format/fixed,
                /pgf/number format/precision=2
            },
            zticklabel={%
                \pgfmathfloatparsenumber{\tick}%
                \pgfmathfloatexp{\pgfmathresult}%
                \pgfmathprintnumber{\pgfmathresult}%
            },
            grid style=dashed,
            legend pos=outer north east,
            legend style={fill=none,font=\footnotesize},
            legend cell align=left,
            view={-45}{8},
        ]
            \addplot3[surf, mesh/rows=4, opacity=0.1, fill opacity=0.3, color=mred] table [x={size}, y={threads}, z={time}, col sep=comma] {data/vector_addition/std-vec.csv};
            \addlegendentry{\stdvec{}}

            \addplot3[surf, mesh/rows=4, opacity=0.1, fill opacity=0.3, color=morange] table [x={size}, y={threads}, z={time}, col sep=comma] {data/vector_addition/rbvec-balanced.csv};
            \addlegendentry{\rbvec{}}

            \addplot3[surf, mesh/rows=4, opacity=0.1, fill opacity=0.3, color=blue] table [x={size}, y={threads}, z={time}, col sep=comma] {data/vector_addition/rrbvec-relaxed.csv};
            \addlegendentry{\rrbvec{} \relaxed{}}

            \addplot3[surf, mesh/rows=4, opacity=0.1, fill opacity=0.3, color=mgreen] table [x={size}, y={threads}, z={time}, col sep=comma] {data/vector_addition/pvec-relaxed.csv};
            \addlegendentry{\pvec{} \relaxed{}}
        \end{loglogaxis}
    \end{tikzpicture}

    \label{fig:adding-two-vectors-par}
    \caption{Adding elements of two \n{} sized vectors parallelized on \emph{K} number of threads.}
\end{figure}

% > =================================================================================
Given two equally sized vectors of integers, the test function adds values at the corresponding indices and returns a new instance of a vector with results. The benchmark is subdivided into three steps:

\begin{enumerate}
    \item Transform two vectors into a single sequence of value pairs by merging their parallel iterators.
    \item Add values of the emitted value pair into a single integer.
    \item Reduce individual sums into a vector of results.
\end{enumerate}

The setup routine prepares two vectors of integers in the \range{[0, N]} problem size range.
% < =================================================================================

\rbvec{} and \rrbvec{} show nearly identical results in sequential benchmarks. This is expected, as append and split operations that create relaxed nodes were not used. Hence, \rrbvec{} remains balanced throughout the test, and is backed by the same representation of \rrbtree{} as \rbvec{}. Both variants are consistently slower compared to \stdvec{}, with a difference of 3.2-3.5 on average.

When executed in parallel, \rrbvec{} starts outperforming \rbvec{} at the size of 1024 elements. The reason why difference becomes apparent after surpassing that size is that the concatenation algorithm used in this project produces balanced \rbtree{} when the height of the tree does not exceed two levels. With the branching factor of 32, the capacity of the tree of two levels is equal to 1024.

As the vector size grows, Rayon performs more slices to achieve optimal vector size per a single thread. This, in turn, results in a higher number of concatenations necessary to combine execution results. Since \rbvec{} has the naive implementation of append and split, its performance degrades with the input size growth. The difference in execution time, depending on size falls into the 1.0-2.3 range.

To keep available threads busy, Rayon divides the available pool of work into smaller pieces. Hence, the growing number of threads increases the performance gap between \rbvec{} and \rrbvec{} even further, as split and append are used more frequently. In the test with 2, 4 and 8 threads, \rbvec{} is slower than \rrbvec{} by a factor of 1.0-2.3, 1.0-2.8, and 1.1-2.12 correspondingly.

Even though \rrbvec{}'s append and split operations are faster for the large-sized vectors, \stdvec{} showed the best results in all tests. It is important to keep in mind that appends and splits constitute only a small number of all operations used in this test. Operations such as push and get, which were extensively used in this benchmark, are still faster for \stdvec{}. Thus, the closest runner up -- \pvec{}, is slower by 1.8-1.9 and 1.6-1.7 in the sequential and 4-threaded parallel benchmarks correspondingly.

\paragraph{Effect of parallelism}
% TODO: this is not entirely true, because 4 threads benchmark runs slightly faster for large datasets compared to sequential bench.
The sequential variant of the benchmark outperformed all subsequent parallel tests. This can be explained by the overhead induced by the distribution of work between multiple threads, which outweighs the benefits of solving a relatively simple problem in parallel.

\subsection{Check if a word is a palindrome}

\begin{figure}[!htbp]

    \center
    \begin{tikzpicture}
        \tikzstyle{every node}=[
            font=\footnotesize,
            inner sep=2pt,
            outer sep=0pt
        ]

        \begin{loglogaxis}[
            smooth,
            width=300pt,
            title style={align=center},
            title={Filtering palindromes on \emph{K} number of threads.},
            ymajorgrids=true,
            xmajorgrids=true,
            zmajorgrids=true,
            xlabel={Vector size (log scale)},
            ylabel={Number of threads (log scale)},
            zlabel={Mean time (log scale) [\millis{}]},
            xticklabels={10, 100, \kilo{1}, \kilo{10}, \kilo{100}, \kilo{400}},
            ytick={1, 2, 4, 8},
            yticklabels={1, 2, 4, 8},
            zticklabel style={
                /pgf/number format/fixed,
                /pgf/number format/precision=2
            },
            zticklabel={%
                \pgfmathfloatparsenumber{\tick}%
                \pgfmathfloatexp{\pgfmathresult}%
                \pgfmathprintnumber{\pgfmathresult}%
            },
            grid style=dashed,
            legend pos=outer north east,
            legend style={fill=none,font=\footnotesize},
            legend cell align=left,
            view={-45}{8},
        ]
            \addplot3[surf, mesh/rows=4, opacity=0.1, fill opacity=0.3, color=mred] table [x={size}, y={threads}, z={time}, col sep=comma] {data/words_filter/std-vec.csv};
            \addlegendentry{\stdvec{}}

            \addplot3[surf, mesh/rows=4, opacity=0.1, fill opacity=0.3, color=morange] table [x={size}, y={threads}, z={time}, col sep=comma] {data/words_filter/rbvec-balanced.csv};
            \addlegendentry{\rbvec{}}

            \addplot3[surf, mesh/rows=4, opacity=0.1, fill opacity=0.3, color=blue] table [x={size}, y={threads}, z={time}, col sep=comma] {data/words_filter/rrbvec-relaxed.csv};
            \addlegendentry{\rrbvec{}}

            \addplot3[surf, mesh/rows=4, opacity=0.1, fill opacity=0.3, color=mgreen] table [x={size}, y={threads}, z={time}, col sep=comma] {data/words_filter/pvec-relaxed.csv};
            \addlegendentry{\pvec{}}
        \end{loglogaxis}
    \end{tikzpicture}

    \label{fig:words-filter}
    \caption{Filtering palindromes on \emph{K} number of threads.}
\end{figure}

% > =================================================================================
The benchmark checks whether a word is a palindrome and annotates it with a boolean flag. As input, we are using a list of English words consisting only of alphabetic characters. The computation stages are the following:

\begin{enumerate}
    \item Transform the given vector of words into a parallel iterator.
    \item Return a tuple of the word and the flag indicating if the word is a palindrome.
    \item Reduce the results to a new instance of a vector of tuples.
\end{enumerate}

\paragraph{The benchmark setup}
The dictionary file contains 370103 words. The benchmark is parameterized over the number of threads and words. The data is loaded into memory only once, and before each run, the setup routine copies \n{} words and passes them to the test as a vector.
% < =================================================================================

As expected, \rbvec{} and \rrbvec{} are equally fast in the sequential test, as both of them are backed by a balanced \rbtree{}. When the benchmark is parallelized, \rrbvec{} gains advantage due to its efficient slice and concatenation operations. The difference between variants grows along with the increasing count of threads. Specifically, it is 1.5-2.0 for the test run with two threads, and 1.7-2.5 for four threads.

The increase in the thread count causes a higher count of vector slices and concatenations. Thus, the bigger the problem size and the thread count is, the more advantages \rrbvec{} provides, demonstrating performance results comparable to \stdvec{}.

\section{Memory benchmarks}
% > =================================================================================
The goals for the memory tests are the following:
\begin{itemize}
    \item Measure and compare the memory footprint of the tree-based and standard vectors.
    \item Evaluate the effectiveness of the structural sharing when cloning a vector.
\end{itemize}

\subsubsection*{Building a vector}
This benchmark is expected to reveal the memory overhead of using a tree instead of the contiguous memory block as its height, and the node count grows. Given the size \n{} as a parameter, this benchmark builds a vector by pushing \n{} values into it. The problem size range is \range{[20, 60k]}.

\subsubsection*{Updating and cloning a vector}
The test builds a new vector of size \n{}, runs a loop from 0 to \n{}, clones a vector, and updates a value at the given index. All cloned instances are accumulated in another vector to observe how well structural sharing helps to save memory. The vector sizes are in \range{[20, 60k]} range.
% < =================================================================================

\chapter{Conclusions and future work}
In this final chapter, I will look at the state and future of the \pvecrs{} project, and how the ideas explored in this thesis can be continued further.

\section{Reflecting on contributions}
This project explores and blends ideas at the intersection of persistent data structures and unique features of Rust to contribute a vector implementation that delivers good performance for all operations, including clone. It makes \pvecrs{} a viable alternative in applications where the fast clone operation is critical.

The list of vectors includes \rbvec{}, \rrbvec{}, and \pvec{}, all of which are based on \rrbtree{}.

\subsection{Balanced vs. relaxed}
The advantage of the relaxed \rbtree{} is the fast appends and splits. \rrbvec{} demonstrates significantly better performance for those operations compared to \rbvec{}, and even outperforms \stdvec{} for the large-sized vectors.

Frameworks for parallelism, such as \rayon{}, take advantage of multiple threads by dividing the work between them. Vectors are subdivided by using the split operation, after which the results are collected back by concatenating them. Therefore, \rrbtree{}'s fast append and split operations are critical for achieving good performance in parallel use cases.

The overhead of relaxation is present in other operations, but it is not significant enough to outweigh the benefits. Also, constraints are relaxed only when append or split is used, meaning that one does not have to pay the cost of the abstraction before using it.

\subsection{Pay only for the features you use}
The project draws inspiration from one of the Rust key features -- zero-cost abstractions. As demonstrated in \Cref{chapter:benchmarks-and-results}, \pvec{} starts as an ordinary, standard vector that delivers great performance for the core operations. When cloned, it employs a technique introduced in this project named \emph{spilling}, which transitions the vector from the flat to the tree-based representation. When transitioned to a balanced \rrbvec{}, \pvec{} offers practically \bigo{1} performance for all operations, including cloning, enabling patterns that extensively rely on copying.

\paragraph{Dynamic representation}
Tree-based vectors are very cheap to clone, but their core operations, even though very fast, do not match the performance of the standard vector due to the nuances of hardware architecture. The dynamic representation aims to offset this cost by using standard vector, switching to the tree-based representation only when cloned.

Results show that dynamic representation effectively improves the performance of all \pvec{} operations, except appending and splitting for large-sized vectors, where tree-based vectors have an advantage. However, since \pvec{} is essentially an additional abstraction layer, it introduces a marginal overhead over the pure variants of its representations -- \stdvec{} and \rrbvec{}.

\paragraph{Unique access}
In the paper Improving RRB-Tree Performance through Transience \cite{improving-performance-through-transience}, the author mentions that the correct use of transient data types can be checked during compile-time in languages that use linear types \cite{linear-types-can-change-the-world}, such as Rust.

This project implements unique access optimization, that is somewhat similar to transience, but is not entirely the same. For example, transients in Clojure\footnote{\url{https://clojure.org/reference/transients}} are created by calling a special function -- \emph{transient}, and converted back to persistent using \emph{persistent!}. Transient types are also thread-local in Clojure, meaning that they cannot be modified outside of the thread where the transient was created.

In Rust, a vector can be considered transient when it is accessed through a mutable reference without calling a special function such as \emph{transient}. The Rust's compiler ensures that the mutable reference is \emph{unique} and that the operation is safe. With that knowledge, the program can proceed to update vector in-place without creating copies -- transitively. Rust also allows moving objects between threads, so a vector instantiated on one thread can be updated on another.

The unique access optimization ensures that the data structure can be mutated in-place only when it is safe. If a vector is cloned before being updated, copy-on-write semantics of \rc{} will enforce path-copying leaving the original untouched.

Benchmarks for mutative operations, such as push, pop, and update that included tests with and without clone showed that updating a data structure in-place was noticeably faster.

Additionally, a "mutable" interface for \pvec{} that makes unique access optimization possible, also makes the API of \pvec{} identical to \stdvec{} and conventional to Rust.

\paragraph{Idiomatic, ergonomic Rust interface}
The idiomatic and convenient interface of \pvecrs{}, identical to the standard one, simplifies the library's integration into existing codebases. A side effect of this design is that both types of vectors can be used interchangeably in a generic manner. For example, the vector type can be substituted during compile-time using feature flags without changing the source code.

Thread-safety is also an optional feature that can be enabled when compiling. This way, developers do not have to pay the cost of using the parallel vector features in sequential applications.

\section{Implementation state}
While the \pvecrs{} core outlined in the thesis is complete, some features and optimizations were left out of the scope. This section describes features and ideas that can be explored further.

\subsection{Supporting all operations of Vec}
The API surface of the \pvecrs{} does not expose the same set of methods as the standard vector does. Available methods are listed in \Cref{tab:vector-apis}.

Having efficient appending and splitting allows us to implement several other operations, such as inserting or deleting an element at any index. The complexity of these operations is bound by the complexity of the discrete operations used to implement them. Thus, uniform performance characteristics across core operations are critical for achieving good all-around performance for the general-purpose vector.

For example, element insertion at the given index can be implemented by splitting the vector at the given index, pushing a new element into the left sub-vector, and then concatenating two sub-vectors back together.

Therefore, the operations that can be implemented by combining or re-using core operations were intentionally left as future work due to the time constraints.

\subsection{Improving the dynamic representation}

\paragraph{Automatically switching to the flat representation}
A distinct feature of \pvec{} is the ability to start as a standard vector and then switch to \rrbvec{} when cloned. However, there is no mechanism to switch back to the flat representation, for example, when all cloned instances are destroyed.

One way to achieve this is by flattening the \rrbtree{} into a standard vector when the underlying tree is not shared with any clones. One could use Rust's destructors to observe when \pvec{} clones go out of the scope. The challenge is to be able to say when the tree is not shared anymore. A brute force approach would be traversing the tree and counting references, but obviously, it is very in-efficient.

An annotated example of this optimization in use is provided in \Cref{lst:switching-to-flat}.

\begin{listing}[H]

    \centering
    \begin{minted}[
        fontsize=\small,
        stripnl=false,
        framesep=4mm,
        frame=lines,
        autogobble,
        linenos
    ]{rust}
        let mut vec_1 = PVec::new();
        // ^ start as a standard Vec

        for i in 0..512 {
            vec_1.push(i);
        }

        vec_1 = vec_1.clone();
        // ^ force switch to RrbVec

        let vec_2 = vec_1.clone();
        { // <-- moving vec_2 to the new scope
            vec_2
        } // <-- vec_2 goes out of the scope and is
          // destroyed, vec_1 switches back to standard Vec

        // execution continues
    \end{minted}

    \caption{An example of switching back to the flat representation}
    \label{lst:switching-to-flat}
\end{listing}

\paragraph{Starting as an array allocated on the stack}
The flat vector representation is efficient because of its cache-friendly memory layout. Since the vector size is not known at the compile-time, it is allocated on the heap. In comparison to the stack, heap allocation is more complex and expensive as it requires the memory allocator to track and manage allocated blocks of memory. Additionally, heap-allocated objects are more likely to cause cache invalidation, as CPU will have to reach a memory segment that potentially is located far outside of its caches.

In an attempt to improve the cache locality properties of the standard vector, authors of the \crate{smallvec}\footnote{\url{https://docs.rs/smallvec/1.2.0/smallvec/}} library introduced a vector implementation that stores a certain number of elements on the stack, and falls back to the heap for larger sizes.

The dynamic representation can be extended with the new representation type that allocates vector on the stack. The vector first will be allocated on the stack, then spill to the heap when exceeding a certain size threshold, and switched to \rrbvec{} when cloned.

One has to be cautious in implementing this optimization. Internally, \pvec{} is backed by the \type{Representation} enum, and in Rust, the enum size is bounded by its largest variant. The variant that holds the stack-allocated buffer will quickly supersede \type{Flat} and \type{Tree} representations if set to be sufficiently large, increasing the overall memory footprint of \pvec{}.

\subsection{Focus and display optimizations}
The notion of \emph{focus} was introduced in Scala's immutable vector implementation and was further studied in \cite{rrb-vector-practical-general-purpose-im-sequence}. Instead of keeping track only of the vector \emph{tail}, the focus is generalized to work with the leaf node, which was last modified. The basis for this is the principle of spatial locality, a heuristic that assumes that collocated elements are more likely to be accessed one after another.

Since the vector has to be thread-safe, focus either has to be modified when the vector itself is modified or to be protected from the concurrent access. The latter comes at additional performance and maintenance costs.

\emph{Display} is a way to keep track of the entire tree branch, from the root to a leaf, where a leaf is the \emph{focused} node. Introducing display to \rrbtree{} requires additional coordination when the tree is modified.

\paragraph{Limitations of Rust}
The strict ownership and borrowing rules introduce additional complexity in implementing the \emph{display}. It is forbidden to acquire and keep mutable references to the node and its children simultaneously. That is a necessary property for display, which essentially is a stack of pointers to nodes that form a path from the root to the leaf nodes.

Alternatively, rather than keeping a stack of mutable pointers, one could use \rc{}. The side effect of this choice is that ownership of \rc{} demands to clone. This, in turn, increments the reference count. When the reference count is bigger than one, any attempts to acquire a mutable pointer will result in the clone of the underlying value. Since display and focus are updated only when the vector itself is modified, it will result in path-copying every time.

The second option is to use the interior mutability pattern in Rust, or \refcell{}. \refcell{} is a container that enforces compile-time rules of the borrow checker at runtime. Offsetting these checks helps to implement the display, but also adds overhead to every other operation, as all tree nodes have to be decorated with \refcell{}.

Even though \emph{display} and \emph{focus} seem to optimize some specific use cases potentially, the additional implementation complexity could cause more bugs and harm performance of other operations making \rrbtree{} less efficient as a general-purpose data structure.

Especially in Rust, the options listed above require either using the unsafe subset of the language features, sacrificing the simplicity, and possibly the reliability and performance. Adding \emph{focus} and \emph{display} to \rrbvec{} is therefore left as future work.

\section{Towards the library of persistent data structures for Rust}
Vector is only one of many other general-purpose data structures provided by the Rust standard library, such as \type{LinkedList}, \type{HashMap}, \type{HashSet}, and others. The ideas discussed in this thesis can be used to implement persistent variants of those data structures. For example, the hash array mapped tries \cite{ideal-hash-trees} can be used as a foundation for \type{HashMap}.

In fact, there are other projects that implement persistent collections for Rust today, such as \imrs{}\footnote{\url{https://docs.rs/im/14.3.0/im/}} and \rpds{}\footnote{\url{https://docs.rs/rpds/0.7.0/rpds/}}. Even though they do not offer the same optimizations and interface as \pvecrs{}, they are a viable alternative for someone who needs other persistent data structures today.

\begin{center}
    \vspace*{6cm}
    \includegraphics[width=8cm, angle=0, trim=10 10 10 10, clip]{images/ferris-waving.png}
\end{center}

\appendix

\chapter{\treerrb{} algorithms}

\section{Rebalancing algorithm}
\begin{listing}[!ht]

    \begin{algorithmic}[1]
        \Function{Rebalance}{left, middle, right}
            \State height \la\ \Call{Max}{left\ts{height},middle\ts{height},right\ts{height}}
            \State root, subtree, node \la\ \Call{CreateNode}{height}

            \For{mergedNode \In\ left + middle + right}
                \If{node\ts{len} = 0 \And\ mergedNode\ts{len} == \m{}}
                    \State \Call{CheckSubtree}{root, subtree}
                    \State subtree[subtree\ts{len}] \la\ mergedNode
                    \State subtree\ts{len}++
                \Else
                    \For{childNode \In\ mergedNode}
                        \If{node\ts{len} = \m{}}
                            \State \Call{CheckSubtree}{root, subtree}
                            \State subtree[subtree\ts{len}] \la\ node
                            \State subtree\ts{len}++
                        \EndIf

                        \State node[node\ts{len}] \la\ childNode
                        \State node\ts{len}++
                    \EndFor
                \EndIf
            \EndFor

            \State \Call{CheckSubtree}{root, subtree}

            \If{node\ts{len} != 0}
                \State subtree[subtree\ts{len}] \la\ node
                \State subtree\ts{len}++
            \EndIf

            \If{subtree\ts{len} != 0}
                \State root[root\ts{len}] \la\ subtree
                \State root\ts{len}++
            \EndIf

            \State \Return root
        \EndFunction

        \State

        \Procedure{CheckSubtree}{root, subtree}
            \If{subtree\ts{len} = \m{}}
                \State root[root\ts{len}] \la\ subtree
                \State root\ts{len}++
            \EndIf
        \EndProcedure
    \end{algorithmic}

    \caption{Rebalancing algorithm for \treerrb{}}
    \label{lst:rrb-tree-rebalance}
\end{listing}

\chapter{Tail optimization for persistent vectors}
\label{sec:tail-optimization}

In practice, changes are often applied to the end or \emph{tail} of the data structure. The stack is designed for such use cases, by offering the \bigo{1} performance for the push and pop operations. Even though \treerrb{} has similar performance characteristics, its push and pop implementations include pesky constant factors in the form of \emph{radix search} and \emph{path copying} algorithms.

The \emph{tail} optimization is intended to offset this cost by reducing the count of the \treerrb{} accesses. Instead of adding or removing elements one by one, changes are batched in the array of size \m{}. This array could be thought of as a leaf node that will be attached to the tree only when it is full.

\section{Optimizing the push operation}
As demonstrated in \Cref{lst:tail-push}, the new value is set into a cloned tail at the tail\ts{size} position. Since the tail is the rightmost leaf node, its size can be used as an index for the new value. If the tail is full, it is pushed into the tree and replaced with an empty tail.

\begin{listing}[!ht]
    \begin{algorithmic}[1]
        \Function{Push}{vec, value}
        \State newTail \la\ \Call{Clone}{vec\ts{tail}}
        \State newTail[tail\ts{size}] \la\ value
        \State newTail\ts{size} \la\ tail\ts{size} + 1
        \State newRoot \la\ vec\ts{root}

        \If{newTail\ts{size} = m}
            \State newRoot \la\ \Call{Push}{vec\ts{root}, newTail}
            \State newTail \la\ \Call{CreateNode}{}
        \EndIf

        \State \Return \Call{CreateVec}{newRoot, newTail}
        \EndFunction
    \end{algorithmic}

    \caption{Tail optimization for persistent vector's push operation}
    \label{lst:tail-push}
\end{listing}

\section{Optimizing the pop operation}
Since a tail might contain values, pop has to remove them first before modifying the \treerrb{}. If the tail is empty, it will be replaced with the rightmost leaf of \treerrb{}. See \Cref{lst:tail-pop} for detailed steps.

\begin{listing}[!ht]

    \begin{algorithmic}[1]
        \Function{Pop}{vec}

        \State newTail \la\ \Call{Clone}{vec\ts{tail}}
        \State newRoot \la\ \nil{}

        \State value \la\ newTail[newTail\ts{size} - 1]
        \State newTail\ts{size} \la\ newTail\ts{size} - 1

        \If{newTail\ts{size} = 0}
            \State newRoot, newTail \la\ \Call{Pop}{vec\ts{root}}
        \Else
            \State newRoot \la\ vec\ts{root}
        \EndIf

        \State \Return \Call{CreateVec}{newRoot, newTail}
        \EndFunction
    \end{algorithmic}

    \caption{Tail optimization for the persistent vector’s pop operation}
    \label{lst:tail-pop}
\end{listing}

\section{Adapting the update and radix search operations}
Changes for both update and radix search are very similar, with the difference that update has to ensure that the original version of vector stays unmodified.

The radix search implementation has to take into account that some of the values can be in the tail. A value is located within the tree if the key is less than the tree size. In this case, the search process is delegated to \treerrb{}. Otherwise, the index for value in the tail is calculated by subtracting the tree size from the key.

\begin{listing}[!ht]

    \begin{algorithmic}[1]
        \Function{Update}{vec, key, value}

        \State root \la\ vec\ts{root}
        \State tail \la\ vec\ts{tail}

        \If{key < root\ts{size}}
            \State newRoot \la\ \Call{Update}{root, key, value}
            \State \Return \Call{CreateVec}{newRoot, tail}
        \Else
            \State newTail \la\ \Call{Clone}{tail}
            \State newTail[key - root\ts{size}] \la\ value
            \State \Return \Call{CreateVec}{root, newTail}
        \EndIf
        \EndFunction
    \end{algorithmic}

    \caption{Using the tail optimization in the update operation}
    \label{lst:tail-update}
\end{listing}

\begin{listing}[!ht]

    \begin{algorithmic}[1]
        \Function{RadixSearch}{vec, key}

        \State root \la\ vec\ts{root}
        \State tail \la\ vec\ts{tail}

        \If{key < root\ts{size}}
            \State \Return \Call{RrbTreeRadixSearch}{root, key}
        \Else
            \State \Return tail[key - root\ts{size}]
        \EndIf
        \EndFunction
    \end{algorithmic}

    \caption{Using the tail optimization in the radix search operation}
    \label{lst:tail-radix-search}
\end{listing}


\cleardoublepage
\pdfbookmark[1]{List of Listings}{bm-listings}
\listoflistings

\cleardoublepage
\pdfbookmark[1]{List of Figures}{bm-figures}
\listoffigures

\cleardoublepage
\pdfbookmark[1]{List of Tables}{bm-tables}
\listoftables

\printbibliography[heading=bibintoc]

\end{document}
